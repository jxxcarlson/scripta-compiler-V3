\documentclass[11pt, oneside]{article}
\usepackage{xcolor}


%% Packages

\usepackage{listings}

%% Standard packages
\usepackage{geometry}
\geometry{letterpaper}
\usepackage{changepage}  % for the adjustwidth environment
\usepackage{graphicx}    % for \includegraphics

%% Index, hyperref
\usepackage[T1]{fontenc}
\usepackage{lmodern}

\usepackage{hyperref}   % load before imakeidx
\usepackage{imakeidx}

%%%%
\usepackage[normalem]{ulem} % for \st
\usepackage{soul}           % for \hl and \sethlcolor
\usepackage{wrapfig}        % for wrapfigure

%% AMS
\usepackage{amssymb}
\usepackage{amsmath}

\usepackage{amscd}

\usepackage{fancyvrb} %% for inline verbatim
\usepackage{makeidx}

%% Chemistry
\usepackage[version=4]{mhchem} % for \ce



%% Commands

\newcommand{\hang}[1]{%
  {%
    \setlength{\leftskip}{1em}%
    \setlength{\hangindent}{1em}%
    \hangafter=1 %
    #1\ \vpace{4}%
  }%
}

\renewcommand{\labelitemi}{\scalebox{0.7}{\textbullet}}

% Dot box = 1em, gap = 1em → total = 2em
\newcommand{\compactItem}[1]{%
  \par
\noindent
  \hangindent=2em \hangafter=1%
  \makebox[1em][l]{\labelitemi}\hspace{1em}#1\par
}

\newcommand{\code}[1]{{\tt #1}}
\newcommand{\ellie}[1]{\href{#1}{Link to Ellie}}
% \newcommand{\image}[3]{\includegraphics[width=3cm]{#1}}

%% width=4truein,keepaspectratio]


% imagecentercaptioned command removed - using standard figure environment instead

\newcommand{\imagecenter}[2]{
   \medskip
   \begin{figure}[htp]
   \centering
    \includegraphics[width=#2]{#1}
    \vglue0pt
    \end{figure}
    \medskip
}

\newcommand{\imagefloat}[4]{
    \begin{wrapfigure}{#4}{#2}
    \includegraphics[width=#2]{#1}
    \caption{#3}
    \end{wrapfigure}
}


\newcommand{\imagefloatright}[3]{
    \begin{wrapfigure}{R}{0.30\textwidth}
    \includegraphics[width=0.30\textwidth]{#1}
    \caption{#2}
    \end{wrapfigure}
}

\newcommand{\hide}[1]{}


\newcommand{\imagefloatleft}[3]{
    \begin{wrapfigure}{L}{0.3-\textwidth}
    \includegraphics[width=0.30\textwidth]{#1}
    \caption{#2}
    \end{wrapfigure}
}
% Font style
\newcommand{\italic}[1]{{\sl #1}}
\newcommand{\strong}[1]{{\bf #1}}
\newcommand{\strike}[1]{\st{#1}}

% Scripta
\newcommand{\ilink}[2]{\href{{https://scripta.io/s/#1}}{#2}}
\newcommand{\markwith}[1]{}
\newcommand{\anchor}[1]{#1}

% Color
\newcommand{\red}[1]{\textcolor{red}{#1}}
\newcommand{\blue}[1]{\textcolor{blue}{#1}}
\newcommand{\violet}[1]{\textcolor{violet}{#1}}
\newcommand{\highlight}[1]{\hl{#1}}
\newcommand{\note}[2]{\textcolor{blue}{#1}{\hl{#1}}}

% WTF?
\newcommand{\remote}[1]{\textcolor{red}{#1}}
\newcommand{\local}[1]{\textcolor{blue}{#1}}

% Unclassified
\newcommand{\subheading}[1]{{\bf #1}\par}
%\newcommand{\term}[1]{{\index{#1}}}
%\newcommand{\termx}[1]{}
\newcommand{\comment}[1]{}
\newcommand{\innertableofcontents}{}


% Special character
\newcommand{\dollarSign}[0]{{\$}}
\newcommand{\backTick}[0]{\`{}}

%% Theorems
\newtheorem{remark}{Remark}
\newtheorem{theorem}{Theorem}
\newtheorem{axiom}{Axiom}
\newtheorem{lemma}{Lemma}
\newtheorem{proposition}{Proposition}
\newtheorem{corollary}{Corollary}
\newtheorem{definition}{Definition}
\newtheorem{example}{Example}
\newtheorem{exercise}{Exercise}
\newtheorem{problem}{Problem}
\newtheorem{exercises}{Exercises}
\newcommand{\bs}[1]{$\backslash$#1}
\newcommand{\texarg}[1]{\{#1\}}


%% Environments
\renewenvironment{quotation}
  {\begin{adjustwidth}{2cm}{} \footnotesize}
  {\end{adjustwidth}}

\def\changemargin#1#2{\list{}{\rightmargin#2\leftmargin#1}\item[]}
\let\endchangemargin=\endlist

\renewenvironment{indent}
  {\begin{adjustwidth}{0.75cm}{}}
  {\end{adjustwidth}}

\newenvironment{box}[1][]{%
  \par\medskip\noindent
  \begin{adjustwidth}{1em}{}
  \ifx\relax#1\relax\else{\bfseries #1}\par\smallskip\fi
}{%
  \end{adjustwidth}\medskip
}

\newcommand{\backtick}{\texttt{\symbol{96}}}

%% NEWCOMMAND

% \definecolor{mypink1}{rgb}{0.858, 0.188, 0.478}
% \definecolor{mypink2}{RGB}{219, 48, 122}
\newcommand{\fontRGB}[4]{
    \definecolor{mycolor}{RGB}{#1, #2, #3}
    \textcolor{mycolor}{#4}
    }

\newcommand{\highlightRGB}[4]{
    \definecolor{mycolor}{RGB}{#1, #2, #3}
    \sethlcolor{mycolor}
    \hl{#4}
     \sethlcolor{yellow}
    }

\newcommand{\gray}[2]{
\definecolor{mygray}{gray}{#1}
\textcolor{mygray}{#2}
}

\newcommand{\white}[1]{\gray{1}[#1]}
\newcommand{\medgray}[1]{\gray{0.5}[#1]}
\newcommand{\black}[1]{\gray{0}[#1]}

% Spacing
\parindent0pt
\parskip5pt

\makeindex[
                          title=Index,
                          columns=2,
                          %% intoc     % include index in the table of contents
                        ]

\begin{document}

\title{Untitled}

\date{}

\author{
test-author
}

\maketitle

\tableofcontents

%%% Line 14
\newcommand{\bp}{{\mathop{\mathbf{p}}}}
\newcommand{\bq}{{\mathop{\mathbf{q}}}}
\newcommand{\br}{{\mathop{\mathbf{r}}}}
\newcommand{\bu}{{\mathop{\mathbf{u}}}}
\newcommand{\bF}{{\mathop{\mathbf{F}}}}
\newcommand{\ta}[1]{\left< #1 \right>}



%%% Line 3


%%% Line 5


%%% Line 14


%%% Line 23
\section{Introduction} \label{introduction}

%%% Line 25
We are going to discuss the   virial theorem, a result
which   relates the average kinetic energy \(\ta{K}\) to the average
potential energy \(\ta{U}\) of a system of particles bound
together by the force of gravity.   By   "bounded", we mean that the system is neither expanding nor contracting. It is in a dynamic equilibrium where the constituent particles are moving
while the system as a whole maintains a kind of structural stability, e.g., it neither collapses nor expands and dissipates.
By "time average," we mean the integral over time on a certain 
interval:

%%% Line 33
\begin{equation}
\ta{Q} = \int_a^b Q(t) dt
\end{equation}

%%% Line 36
The virial theorem has a wide variety of applications.   We give just 
two examples: computing the core temperature of the sun and computing the 
mass of the Coma cluster of galaxies.   In the first case the particles
are fully ionized atoms of hydrogen and helium.   In the second case
they are galaxies.

%%% Line 42
\section{Derivation} \label{derivation}

%%% Line 44
To explain the theorem, consider first the \index{moment of inertia}\textit{moment of inertia} of the syste

%%% Line 46
\begin{equation}
\label{moment-of-inertia}
I = \sum_i m_i \br_i^2
\end{equation}

%%% Line 49
Up to a constant, its derivative is

%%% Line 51
\begin{equation}
\label{moment-of-inertia-deriv}
\frac{1}{2} \dot I =  \sum_i m_i \br_i \cdot \dot \br_i =\sum_i \br_i \cdot \bp_i
\end{equation}

%%% Line 55
where \(\br_i \cdot \bp_i\) is \(|\br_i|\) times the radial 
component of the momentum.   The quantity \(\dot I\) is
therefore a kind of radially weighted average of the
radial components of the momenta of the particles.
This quantity is also a function of the time \(t\).
Thus, if \(\dot I (t) > 0\), the system is expanding, 
whereas if \(\dot I(t) < 0\), it is contracting. If \(I(t) = 0\), the system is \u{stable} at time \(t\).

%%% Line 63
Now consider the second derivative of the moment of inertia:

%%% Line 65
\begin{align}
\frac{1}{2} \ddot I &=  \sum_i m_i \dot \br_i \cdot \dot \br_i +  \sum_i m_i \br_i \cdot \ddot \br_i\\
&= 2\sum_i K_i +  \sum_i \br_i \cdot \bF_i\\
&= 2K -  \sum_i \br_i \cdot GMm_i \frac{\br_i}{ |\br_i|^3}\\
&= 2K -  \sum_i  \frac{GMm_i}{ |\br_i|}\\
&= 2K + U
\end{align}

%%% Line 73
where we have applied Newton's law of graviation and where \(U\) is the gravaitational potential energy. Taking time averages, we have

%%% Line 75
\begin{equation}
\ta{K} + 2\ta{U} = \dot I(b) - \dot I(a)
\end{equation}

%%% Line 78
Thus, for a stable system,

%%% Line 80
\begin{equation}
2\ta{K} + \ta{U} = 0
\end{equation}

%%% Line 83
\section{Applications} \label{applications}

%%% Line 85
\subsection{Mass of Galactic Clusters} \label{mass-of-galactic-clusters}

%%% Line 87
Let \(\sigma\) be the velocity dispersion in a cluster of galaxies,
and let \(M\) be its mass.   Then the average kinetic energy is

%%% Line 90
\begin{equation}
\left< K \right> = \frac{M\sigma^2}{ 2}
\end{equation}

%%% Line 93
We claim that the average potential energy is

%%% Line 95
\begin{equation}
\left< U \right> \approx - \frac{GM^2}{R}
\end{equation}

%%% Line 98
where \(R\) is the radius of the cluster.
Using the virial theorem, we find that the mass of the cluster is

%%% Line 101
\begin{equation}
M \approx \frac{R\sigma^2}{ G}
\end{equation}

%%% Line 105
Typical values of \(\sigma\) are \(500-1500 \text{ km/sec}\).   This
implies

%%% Line 108
\begin{equation}
M_{\text{cluster}} \sim 10^{14}-10^{15} M_\odot.
\end{equation}

%%% Line 111
A cluster typically contains 100 to 1000 galaxies, so the luminosity of the cluster is

%%% Line 113
\begin{equation}
L_\text{cluster} \sim 10^{12} L_\odot
\end{equation}

%%% Line 116
Thus the mass-to-luminosity ratio of a cluster is about 200 to 500 times same ratio for the sun.   \u{Conclusion:} lots of dark matter!

%%% Line 118
\subsection{The Potential Energy Term} \label{the-potential-energy-term}

%%% Line 120
The exact potential energy for a system of particle is

%%% Line 122
\begin{equation}
U = - \sum \frac{Gm_im_j}{r_{ij}}
\end{equation}

%%% Line 125
In the double summation we replace \(r_{ij}\) by \(R\)

%%% Line 128
For a large cluster of mass \(M\), one approximates this by
the typical separation scale \(R\) and \(m_im_j\) by \(m^2\) where
\(m\) is the average galactic mass.   Let \(n\) be the 
number of galaxies.   Then sum is approximately
equal to \((nm)^2 =   M^2\). Up to a scale factor \(\alpha\), we have

%%% Line 134
\begin{equation}
U \approx -\alpha \frac{GM^2}{R}
\end{equation}

%%% Line 137
For a uniform mass distribution, \(\alpha = 3/5\).   For a more
centrally concentrated mass distribution, \(\alpha\) is closer to 1.

%%% Line 140
\subsection{Temperature at the Core of the Sun} \label{temperature-at-the-core-of-the-sun}

%%% Line 142
The internal energy of the sun viewed as a monoatomic gas is

%%% Line 144
\begin{equation}
\ta{K} = \frac{3}{2} NkT
\end{equation}

%%% Line 147
where \(N\) is the number of atoms and \(k\) is Boltzmann's constant.
The Gravitational binding energy is

%%% Line 150
\begin{equation}
\ta{U} = -\alpha \frac{GM^2}{R}
\end{equation}

%%% Line 153
Apply the virial theorem \(\ta{K} = -\ta{U}/2\) to obtain

%%% Line 155
\begin{equation}
\frac{3}{2} NkT = \frac{1}{2} \frac{GM^2}{R}
\end{equation}

%%% Line 158
Solve for the temperature:

%%% Line 160
\begin{equation}
T = \frac{\alpha GM^2}{ 3NkR}
\end{equation}

%%% Line 163
To find the number \(N\), let   \(\mu = 0.6\) be the mean molecular weight (hydrogen-helium mix).
Let \(m_P\) be the mass of a proton.   Then the number of atoms in the sun
is 

%%% Line 167
\begin{equation}
N = \frac{M}{ \mu m_P}
\end{equation}

%%% Line 170
and so finally

%%% Line 172
\begin{equation}
T = \frac{\alpha GM \mu m_p}{ 3kR}
\end{equation}

%%% Line 175
Now plug in the needed constants

%%% Line 177
\begin{align}
G &= 6.67 \times 10^{-11} \text{ SI units}\\
M_\odot &= 1.99 \times 10^{30} \text{ kg}\\
R_\odot &= 6.96 \times 10^n \text{ m}\\
m_P &= 1.67 \times 10^{-27} \text{ kg}\\
k &= 1.38 \times 10^{-23} \text{ J/K}\\
\alpha &\approx 3/5 \text{ (uniform sphere)}
\end{align}

%%% Line 185
to get

%%% Line 187
\begin{equation}
T \approx 2.8 \times 10^6 \text{ K}
\end{equation}

%%% Line 190
Very hot!   Enough to have sustained a nuclear fusion reaction 
for the last four billion years.

%%% Line 193
\section{Appendix} \label{appendix}

%%% Line 195
\subsection{Potential Energy} \label{potential-energy}

%%% Line 200
\section{References} \label{references}

%%% Line 202
\href{https://en.wikipedia.org/wiki/Virial_mass}{Wkipedia reference}

%%% Line 204
\href{https://sites.astro.caltech.edu/~george/ay127/Ay127_GalClusters.pdf}{Caltech Slides} << \textcolor{red}{good!}

%%% Line 206
\href{https://www.astro.umd.edu/~richard/ASTR480/Clusters_lecture2.pdf}{UMD slides}

%%% Line 208
\href{https://phys.libretexts.org/Bookshelves/Astronomy__Cosmology/Supplemental_Modules_(Astronomy_and_Cosmology)/Cosmology/Astrophysics_(Richmond)/18%3A_Using_the_Virial_Theorem_-_Mass_of_a_Cluster_of_Galaxies#:~:text=The%20right%2Dhand%20side%20depends,of%20bodies%20in%20the%20system.}{Physics Libre Reference}

\clearpage

\printindex



\end{document}
