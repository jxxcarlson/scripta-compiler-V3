\documentclass[11pt, oneside]{article}
\usepackage{float}
\usepackage{graphicx}
\usepackage{wrapfig}
\usepackage{xcolor}
\graphicspath{ {image/} }



%% Packages

\usepackage{listings}

%% Standard packages
\usepackage{geometry}
\geometry{letterpaper}
\usepackage{changepage}  % for the adjustwidth environment
\usepackage{graphicx}    % for \includegraphics

%% Index, hyperref
\usepackage[T1]{fontenc}
\usepackage{lmodern}

\usepackage{hyperref}   % load before imakeidx
\usepackage{imakeidx}

%%%%
\usepackage[normalem]{ulem} % for \st
\usepackage{soul}           % for \hl and \sethlcolor
\usepackage{wrapfig}        % for wrapfigure

%% AMS
\usepackage{amssymb}
\usepackage{amsmath}

\usepackage{amscd}

\usepackage{fancyvrb} %% for inline verbatim

%% Chemistry
\usepackage[version=4]{mhchem} % for \ce



%% Commands

\newcommand{\hang}[1]{%
  {%
    \setlength{\leftskip}{1em}%
    \setlength{\hangindent}{1em}%
    \hangafter=1 %
    #1\ \vpace{4}%
  }%
}

\renewcommand{\labelitemi}{\scalebox{0.7}{\textbullet}}

% Dot box = 1em, gap = 1em → total = 2em
\newcommand{\compactItem}[1]{%
  \par
\noindent
  \hangindent=2em \hangafter=1%
  \makebox[1em][l]{\labelitemi}\hspace{1em}#1\par
}

\newcommand{\code}[1]{{\tt #1}}
\newcommand{\ellie}[1]{\href{#1}{Link to Ellie}}
% \newcommand{\image}[3]{\includegraphics[width=3cm]{#1}}

%% width=4truein,keepaspectratio]


% imagecentercaptioned command removed - using standard figure environment instead

\newcommand{\imagecenter}[2]{
   \medskip
   \begin{figure}[htp]
   \centering
    \includegraphics[width=#2]{#1}
    \vglue0pt
    \end{figure}
    \medskip
}

\newcommand{\imagefloat}[4]{
    \begin{wrapfigure}{#4}{#2}
    \includegraphics[width=#2]{#1}
    \caption{#3}
    \end{wrapfigure}
}


\newcommand{\imagefloatright}[3]{
    \begin{wrapfigure}{R}{0.30\textwidth}
    \includegraphics[width=0.30\textwidth]{#1}
    \caption{#2}
    \end{wrapfigure}
}

\newcommand{\hide}[1]{}


\newcommand{\imagefloatleft}[3]{
    \begin{wrapfigure}{L}{0.3-\textwidth}
    \includegraphics[width=0.30\textwidth]{#1}
    \caption{#2}
    \end{wrapfigure}
}
% Font style
\newcommand{\italic}[1]{{\sl #1}}
\newcommand{\strong}[1]{{\bf #1}}
\newcommand{\strike}[1]{\st{#1}}

% Scripta
\newcommand{\ilink}[2]{\href{{https://scripta.io/s/#1}}{#2}}
\newcommand{\markwith}[1]{}
\newcommand{\anchor}[1]{#1}

% Color
\newcommand{\red}[1]{\textcolor{red}{#1}}
\newcommand{\blue}[1]{\textcolor{blue}{#1}}
\newcommand{\violet}[1]{\textcolor{violet}{#1}}
\newcommand{\highlight}[1]{\hl{#1}}
\newcommand{\note}[2]{\textcolor{blue}{#1}{\hl{#1}}}

% WTF?
\newcommand{\remote}[1]{\textcolor{red}{#1}}
\newcommand{\local}[1]{\textcolor{blue}{#1}}

% Unclassified
\newcommand{\subheading}[1]{{\bf #1}\par}
%\newcommand{\term}[1]{{\index{#1}}}
%\newcommand{\termx}[1]{}
\newcommand{\comment}[1]{}
\newcommand{\innertableofcontents}{}


% Special character
\newcommand{\dollarSign}[0]{{\$}}
\newcommand{\backTick}[0]{\`{}}

%% Theorems
\newtheorem{remark}{Remark}
\newtheorem{theorem}{Theorem}
\newtheorem{axiom}{Axiom}
\newtheorem{lemma}{Lemma}
\newtheorem{proposition}{Proposition}
\newtheorem{corollary}{Corollary}
\newtheorem{definition}{Definition}
\newtheorem{example}{Example}
\newtheorem{exercise}{Exercise}
\newtheorem{problem}{Problem}
\newtheorem{exercises}{Exercises}
\newcommand{\bs}[1]{$\backslash$#1}
\newcommand{\texarg}[1]{\{#1\}}


%% Environments
\renewenvironment{quotation}
  {\begin{adjustwidth}{2cm}{} \footnotesize}
  {\end{adjustwidth}}

\def\changemargin#1#2{\list{}{\rightmargin#2\leftmargin#1}\item[]}
\let\endchangemargin=\endlist

\renewenvironment{indent}
  {\begin{adjustwidth}{0.75cm}{}}
  {\end{adjustwidth}}


%% NEWCOMMAND

% \definecolor{mypink1}{rgb}{0.858, 0.188, 0.478}
% \definecolor{mypink2}{RGB}{219, 48, 122}
\newcommand{\fontRGB}[4]{
    \definecolor{mycolor}{RGB}{#1, #2, #3}
    \textcolor{mycolor}{#4}
    }

\newcommand{\highlightRGB}[4]{
    \definecolor{mycolor}{RGB}{#1, #2, #3}
    \sethlcolor{mycolor}
    \hl{#4}
     \sethlcolor{yellow}
    }

\newcommand{\gray}[2]{
\definecolor{mygray}{gray}{#1}
\textcolor{mygray}{#2}
}

\newcommand{\white}[1]{\gray{1}[#1]}
\newcommand{\medgray}[1]{\gray{0.5}[#1]}
\newcommand{\black}[1]{\gray{0}[#1]}

% Spacing
\parindent0pt
\parskip5pt

\makeindex[
                          title=Index,
                          columns=2,
                          %% intoc     % include index in the table of contents
                        ]

\begin{document}

\title{Untitled}

\date{}

\author{
test-author
}

\maketitle

\tableofcontents

%%% Line 371
\newcommand{\pd}[2]{\frac{\partial #1}{ \partial #2}}
\newcommand{\pdd}[2]{\frac{\partial^2 #1}{ \partial #2^2}}



%%% Line 3


%%% Line 6
\begin{center}
\textcolor{red}{(( D R A F T ))}
\end{center}

%%% Line 9
\par\par

%%% Line 11
\begin{indent}
\textit{Scripta is now running a draft of the new Scripta compiler. It is a work in progress, so some things might be broken or just not look quite right. The final, polished version is coming soon.}
\end{indent}

%%% Line 14
\section{What is Scripta?} \label{what-is-scripta}

%%% Line 16
Scripta is an app for creating, editing, and publishing articles with beautiful things like this

%%% Line 18
\begin{center}
\includegraphics[width=0.51\textwidth]{https://imagedelivery.net/9U-0Y4sEzXlO6BXzTnQnYQ/83fdbf6c-79d5-44e7-6ac6-00cdc7785000/public}
\end{center}

%%% Line 21
and this

%%% Line 23
\begin{equation}
\pdd{u}{x} + \pdd{u}{y} + \pdd{u}{z} = \frac{1}{c^2} \pdd{u}{t}
\end{equation}

%%% Line 26
\textit{The equation above governs the transmission of light from sun to flower to the eye of the beholder as in Figure 1} .

%%% Line 28
\textbf{What we aim to do.} (1) Provide a tool that makes it easy to write short notes with images and equations --- for yourself, your friends and colleagues, or for the class you are teaching. Becuase Scripta.io is a web-based platform, it is easy to "publish" your writing. One way to do this is to provide a link to your document. As an example, the link for this document is

%%% Line 30
\begin{indent}
https://scripta.io/g/jxxcarlson:welcome-to-scripta.1
\end{indent}

%%% Line 33
Paste it into your browser and you will see that it works. The \lstinline|/g/| means that anyone using this link is automatically signed in as a guest --- a user with limited privileges A guest cannot, for example create documents.

%%% Line 35
(2) Provide what you need to write longer work: multi-section articles and books.

%%% Line 37
(3) Any Scripta document can be printed to PDF (see the printer icon) and can be exported to LaTeX. \textcolor{gray}{By exporting your document and adding custom style files, you can satisfy the requirements of your publisher.}

%%% Line 39
\textbf{Example:} Here is the \href{https://pdfserv.app/pdf/jxxcarlson-id-c60a3d8c-ad17-4770-adfd-54080dc5689c.pdf}{PDF file} for this document.

%%% Line 41
\subsection{The Scripta Language} \label{the-scripta-language}

%%% Line 43
The scripta language consists of just two things, \index{elements} and \index{blocks} . \footnote{Our aim is to design a langauge that is as simple as possible without losing expressive power.}

%%% Line 45
\subsubsection{Elements} \label{elements}

%%% Line 47
Ordinary text, like "Happy New Year" is an element. So is \lstinline|[i Happy New Year]| . This is an \index{italic element} . It renders as \textit{Happy New Year} . The \index{bold element} looks like \lstinline|[b Happy New Year]| . I renders as you would expect, \textbf{Happy New Year} . Thus \lstinline|[i [b Happy New Year]]| yields \textit{\textbf{ Happy New Year}} . You can also do things like \lstinline|[red [i Happy New Year]]| to get \textcolor{red}{\textit{ Happy New Year}} .

%%% Line 49
You can write mathematical formulas, equations, etc. using a \index{math element} , e.g., $a^2 + b^2 = c^2$ . Here is what we wrote: \lstinline|[math a^2 + b^2 = c^2]| . The shorthand form \lstinline|[m a^2 + b^2 = c^2]| also works. If you want, you can use TeX/LaTex syntax, \lstinline|a^2 + b^2 = c^2| .

%%% Line 51
For code, you can use a \index{code element} . Thus \lstinline|a^2 + b^ = c^2| is produced by writing \lstinline|[code a^2 + b^ = c^2]| . You can also enclose code in backticks, e.g., ` a^2 + b^2 = c^2 ` .

%%% Line 53
\subsubsection{Other elements} \label{other-elements}

%%% Line 55
Below are some elements of note. There are many more --- for a complete list and a fuller explanation, see XXX.

%%% Line 57
\compactItem{\lstinline|[strike this text]| $\to$ \sout{this text} .}
\compactItem{\lstinline|[highlight this text]| $\to$ \highlight{this text}}
\compactItem{\lstinline|[index atom]| adds "atom" to the document index. Use an index block to display the index.}
\compactItem{Use \lstinline|[cite einstein1905a]| to cite the reference einstein1905a in the bibliography. Use a bibref block to make the actual reference. Example: In 1905, Albert Einstein published three ground-breaking papers: \cite{einstein1905a} , \cite{einstein1905b} , \cite{einstein1905c} . The first was instrumental in the development of quantum mechanics, the second in the atomic theory of matter. The third introduce the special theory of relativity and the famous equation $E = mc^2$ .}

%%% Line 62
\subsubsection{Blocks} \label{blocks}

%%% Line 64
\begin{figure}[h]
  \centering
  \includegraphics[width=0.11666666666666667\textwidth]{https://imagedelivery.net/9U-0Y4sEzXlO6BXzTnQnYQ/5c12dfcf-0698-4e0b-733b-f9abf2bfff00/public}
  \caption{Jellyfish}
  \label{fig:jellyfish}
\end{figure}

%%% Line 67
\begin{paragraph}
The tiny image of jellyfish that you see on your left is put there using an \index{image block} . Click on the image to expand it. Click on the expanded image to return the view you ser here. Image blocks have a lot of options. Lets begin with the simplest case, no option at all:
\end{paragraph}

%%% Line 70
\begin{verbatim}
| image
https://imagedelivery.net/9U-0Y4s...
\end{verbatim}

%%% Line 74
It has the form

%%% Line 76
\begin{verbatim}
HEADER
BODY
\end{verbatim}

%%% Line 80
In this case, the header is the single line \lstinline|| image| . The body is the line \lstinline|https://imagedelivery.net...| --- the "internet address" of the image.

%%% Line 82
The image block we used above is a little more complex:

%%% Line 84
\begin{verbatim}
| image expandable 
| float:left 
| width:70 
| caption:Jellyfish
https://imagedelivery.net/9U-0Y4s...
\end{verbatim}

%%% Line 91
In this case the header consists of four lines, each beginning with the pipe symbol "|" followed by a space. The first line of the header is "| image expandable". Here "expandable" is an \index{argument} . These are single words that change the behavior of the image block. In the case at hand, it means that if you click on the image it pops to fill a much larger screen. Clicking the expanded image returns it to its normal state. There can be more than one argument, but these always come before the properties.

%%% Line 93
The next line of the header is the text "float:left". It defines a \index{property} of the image block. Properties are defined by text of the form \lstinline|key:value| . In this case the key is "float" and the value associated with that key is "left". A block with this property hugs the left margin of the document and the text below it flows around it. There is also a \lstinline|float:right| property. The properties set by the next two lines define the width in pixels of image displayed and its caption.

%%% Line 95
\subsubsection{Math Blocks
   Here is a very simple \index{math block} :} \label{math-blocks-here-is-a-very-simple-indexmath-block-}

%%% Line 98
\begin{verbatim}
| math
a^2 + b^2 = c^2
\end{verbatim}

%%% Line 102
It yields this:

%%% Line 104
\begin{equation}
a^2 + b^2 = c^2
\end{equation}

%%% Line 107
We can add a \index{label} property to it:

%%% Line 109
\begin{verbatim}
| math label:pythag
a^2 + b^2 = c^2
\end{verbatim}

%%% Line 113
\begin{equation}
\label{pythag}
a^2 + b^2 = c^2
\end{equation}

%%% Line 116
Any math block that has the label property is automatically numbered. The label can be used with the \index{mathref} element to refer back to the equation as in the paragraph below:

%%% Line 118
\begin{indent}
Equation \eqref{pythag} is the \textit{Pythagorean theorem} , where $a$ and $b$ are the lengths of the legs of a right triangle, and $c$ is the length of the hypotenuse.
\end{indent}

%%% Line 121
Click on the reference one to go to equation (1). Press ESC to return to where you were. The mathref element we used here is \lstinline|[mathref pythag]| .

%%% Line 123
\textit{For LaTeX users:} As a convenience, Scripta provides \index{equation blocks} and \index{eqref elements} . These are synomyms for math blocks and mathref elements. They corresond to equation environments and eqref macros in LaTeX.

%%% Line 125
\subsubsection{Chemistry Blocks} \label{chemistry-blocks}

%%% Line 127
Chemical formulas and reactions can be rendered using \index{chem elements} and \index{chem blocks} . The formula for methane is $\ce{CH4}$ . Burning hydrogen in oxygen takes place fia the reaction $\ce{2H2 + O2 -> 2H2O + Energy}$ .

%%% Line 129
To produce the text above, we used

%%% Line 131
\begin{verbatim}
[chem CH4] and [chem 2H2 + O2 -> 2H2O + Energy]
\end{verbatim}

%%% Line 134
See \href{https://texdoc.org/serve/mhchem/0}{The mhchem bundle by Martin Hensel} for a comprehensive discussion of what you can do with chem elements and blocks. These blocks relie on mhchem.

%%% Line 136
\subsubsection{Code Blocks} \label{code-blocks}

%%% Line 138
Below is an example of a \index{code block}

%%% Line 140
\begin{verbatim}
  def is_prime(n):
      if n < 2:
          return False
      for k in range(2, int(n**0.5) + 1):
          if n % k == 0:
              return False
      return True
  
  
  def print_primes_up_to(N):
      for p in range(2, N + 1):
          if is_prime(p):
              print(p)
  
  
  # example usage
  N = 100
  print_primes_up_to(N)
\end{verbatim}

%%% Line 160
XXX: Add discussoin of multilne blocks. XXX

%%% Line 162
\subsubsection{Other blocks} \label{other-blocks}

%%% Line 164
Below is a list of frequently used blocks. For a complete list or for more details, see XXX.

%%% Line 166
\compactItem{theorem and its relatives (lemma, definition, etc.)}
\compactItem{link \lstinline|[link NYT https:nytimes.com]| $\to$ \href{https:nytimes.com}{NYT} is a link to the New York Times.}
\compactItem{ilink:}
\compactItem{quotation}
\compactItem{table, csvtable}
\compactItem{index: makes an index of the words or phrases you have marked with \lstinline|[index ...]| .}
\compactItem{references??}
\compactItem{cite, bibitem}
\compactItem{etc.}

%%% Line 176
\subsection{The Scripta App} \label{the-scripta-app}

%%% Line 178
\begin{center}
\includegraphics[width=0.3333333333333333\textwidth]{https://imagedelivery.net/9U-0Y4sEzXlO6BXzTnQnYQ/f3c5c902-809c-4629-7015-a4327103a000/public}
\end{center}

%%% Line 181
On the left you see a thumbnail preview of the Scripta.io app. This document is being edited. You see the source text on the left, the rendered text on the right. If you click on this image, it will expand to show you a much larger view of the preview. Click on it again to restore the previous view.

%%% Line 183
Do this now. Do it several times to ingrain it in your muscle memory.

%%% Line 185
One of the main features of Scripta is that as you type new source text or make edits to the source text, those changes will be rendered almost instantly in the rendered text window. See (XXX VIDEO)

%%% Line 187
To edit a document, click on the Pen icon which you will see above the rendered text. You can also click on any part of the rendered text. \footnote{There are some exceptions, e.g. ilink elements XXX.} Click on the $\boxtimes$ (XXX) to leave the editor.

%%% Line 189
There is a lot more to the Scripta app. See XXX. You can, for example, export Scripta documents to PDF and then print it (XXX SEE VIDEO). Just click on the printer icon and follow directions. You can also export any Scripta document to LaTeX. \footnote{There are some features XXX that are not compatible with LaTeX, so these are not exported. See XXX for a complete list of these.} \textcolor{gray}{By adding style files, you can prepare it for publication by a journal that requires submission in LaTeX in a particular format. XXX: to be investigated}

%%% Line 191


%%% Line 197
\section{Dive Right In} \label{dive-right-in}

%%% Line 199
\textit{You don't have to read all of this article or even very much of it to use Scripta. Take a look at this short video (XXX) and then get started writing. The video will show you all you need to know. You can come back to this manual as needed.}

%%% Line 201
\section{Features of Scripta.io} \label{features-of-scriptaio}

\begin{itemize}

%%% Line 203
\item \textbf{No set-up} . Create a new document with the click of a button. Then start typing.

%%% Line 205
\item \textbf{Web-based} . Scripta produces web pages, not PDFs, although you can do that too.

%%% Line 207
\item \textbf{Instant rendering} . As you type, the web pageyou are working on --- text, images, equations, cross references, table of contents, etc. --- is updated. All in real time.

%%% Line 209
\item \textbf{Modern editor} . Click on a word in the rendered text.It will be highlighted in the source text. Selectsome source text, type ctrl-S. The corresponding rendered text will be highlighted. Type ctrl-1 to remove thehighlight.

%%% Line 211
\item \textbf{Available everywhere, anytime} . Your writing is available to you, your students, and your colleages anywhere there is an internet connection. Readable on smart phones, tablets, laptops, and desktop computers.

%%% Line 213
\item \textbf{No vendor lock-in} . Philosophically, we are opposed to vendor lock-in. It is wrong. Practically, we provide tools for you to export both individual documents and yourentire corpus of documents. See XXX. Additionally,the Scripta compiler, which is open source, is avaiable on \href{https://github.com/jxxcarlson/scripta-compiler-v2}{Github} .

%%% Line 215
\item \textbf{Integrated image uploader} . Click on thecamera icon for placing images in your document. Paste the image into the uploader, then paste the resulting image block in the editor.

\end{itemize}

%%% Line 217
\begin{center}
\includegraphics[width=0.51\textwidth]{https://imagedelivery.net/9U-0Y4sEzXlO6BXzTnQnYQ/5111dd45-daa1-4183-315a-7bec8802a700/public}
\end{center}

%%% Line 220
\section{Examples} \label{examples}

\begin{itemize}

%%% Line 222
\item \href{https://scripta.io/s/id-19124175-eb43-4d9d-9f7b-7688f1ee8e1c}{Graphs and Colorings}

%%% Line 224
\item \href{https://scripta.io/s/jxxcarlson:science-stories.1}{Blog}

%%% Line 226
\item \href{https://scripta.io/s/jxxcarlson:physics-notebook}{Physics Notes}

%%% Line 228
\item \href{https://scripta.io/s/jxxcarlson:abstract-art}{Abstract Art}

\end{itemize}

%%% Line 230
See \href{https://scripta.io/s/jxxcarlson:welcome-to-scripta.1#2165d672-ed0e-49ad-a345-705f036d1403}{the discussion below} for an explanation of the differences between the way mathematics is written in Scripta and the way it is written in LaTeX. Despite these differences, Scripta documents can always be \href{https://scripta.io/s/jxxcarlson:welcome-to-scripta.1#ecebaadd-7cbb-4bb3-b184-0209bd4af583}{rendered as PDF} , and they can be \href{https://scripta.io/s/jxxcarlson:welcome-to-scripta.1#3b122d77-ed3a-4532-81a1-711186ce9168}{exported to LaTeX} .

%%% Line 232
We could have written \lstinline|| equation numbered| if we wanted the equation to be automatically numbered. We could have said \lstinline|| equation label:wave-equation| to label the equation. A labeled equation is numbered:

%%% Line 234
\begin{equation}
\label{wave-equation}
\pdd{u}{x} + \pdd{u}{y} + \pdd{u}{z} = \frac{1}{c^2} \pdd{u}{t}\qquad\text{Wave Equation}
\end{equation}

%%% Line 239
We can refer to it using the element \lstinline|[eqref wave-equation]| :

%%% Line 241
\begin{indent}
The equation \eqref{wave-equation} governs many phenomena, e.g. both sound and light waves.
\end{indent}

%%% Line 244
Blocks in Scripta are the functional equivalent of environments in LaTeX. Because a block is indexinated by an empty line, errors coming from badly indexinated blocks, as in LaTeX, are impossible.

%%% Line 246
\subsection{Blank lines in blocks} \label{blank-lines-in-blocks}

%%% Line 248
If you do need blank lines in a block, just indent the body of the block, like this:

%%% Line 250
\begin{verbatim}
| code
  a := 1
  b := 2
  
  c := a + b
\end{verbatim}

%%% Line 257
The result is this:

%%% Line 259
\begin{verbatim}
  a := 1
  b := 2
  
  c := a + b
\end{verbatim}

%%% Line 265
\subsection{Ergonomic TeX (ETeX)} \label{ergonomic-tex-etex}

%%% Line 267
\markwith{2165d672-ed0e-49ad-a345-705f036d1403} This is ETeX, or \quote{Ergonomic TeX} . The idea is to reduce visual noise by eliminating backslashes and curly braces as much as possible. Thus, instead of \lstinline|\\alpha^2| , we write simply \lstinline|alpha^2| . \footnote{Note use of "mark" in this paragraph. This use needs to be documented in the manual (and improved).} The first and second partial derivatives were defined in a \textit{macro} block:

%%% Line 269
\begin{verbatim}
| mathmacros
pd:       frac(partial #1, partial #2)
pdd:      frac(partial^2 #1, partial #2^2)
\end{verbatim}

%%% Line 274
The usual LaTeX way is

%%% Line 276
\begin{verbatim}
\newcommand\{\pd\}[2]\{\frac\{\partial #1\}\{\partial #2\}\}
\newcommand\{\ppd\}[2]\{\frac\{\partial^2 #1\}\{\partial #2^2\}\}
\end{verbatim}

%%% Line 281
\subsection{Exceptional Elements} \label{exceptional-elements}

%%% Line 283
Some elements have an off-beat syntax. To write code in-line, as in \lstinline|a := 2| , write

%%% Line 285
\begin{verbatim}
... as in `a := 2` ....
\end{verbatim}

%%% Line 288
To write math inline, as in $(a_1 + a_2)^3$ , write

%%% Line 290
\begin{verbatim}
as in $(a_1 + a_2)^3$, write
\end{verbatim}

%%% Line 293
\subsection{Exceptional Blocks} \label{exceptional-blocks}

%%% Line 295
Some blocks have an off-beat syntax. For sections, subsections, etc., we borrow from Markdownk using hashmarks \lstinline|#| . The skeleton of an article might be

%%% Line 297
\begin{verbatim}
  | title
  The Kingdoms of Life
  
  # Archaea
  
  # Bacteria
  
    ## Proteobacteria
    
    ## Cyanobacteria
  
  # Protista
  
  # Fungi
  
  # Plantae
\end{verbatim}

%%% Line 315
Itemized and numbered lists are also exceptional blocks, again inspired by Markdown. Here is an itemized list:

%%% Line 317
\begin{verbatim}
  - Milk
  
  - Cookies
  
  - Paper plates
\end{verbatim}

%%% Line 324
It renders as

\begin{itemize}

%%% Line 326
\item Milk

%%% Line 328
\item Cookies

%%% Line 330
\item Paper plates

\end{itemize}

%%% Line 332
Numbered lists are written as

%%% Line 334
\begin{verbatim}
  . Milk
  
  . Cookies
  
  . Paper plates
\end{verbatim}

%%% Line 341
Itrenders like this:

\begin{enumerate}

%%% Line 343
\item Milk

%%% Line 345
\item Cookies

%%% Line 347
\item Paper plates

\end{enumerate}

%%% Line 350
\section{Ergonomic} \label{ergonomic}

%%% Line 352
Scripta is desiged to be ergonomic. When you write mathematical text, for example,

%%% Line 354
\begin{equation}
\frac{\sin \theta_1}{ \sin \theta_2} = \frac{v_1}{ v_2}
\end{equation}

%%% Line 357
you write

%%% Line 359
\begin{verbatim}
| equation numbered
frac(sin theta_1, sin theta_2) = frac(v_1, v_2)
\end{verbatim}

%%% Line 363
instead of the usal TeX:

%%% Line 365
\begin{verbatim}
| equation
\frac\{\sin\theta_1\}\{\sin\theta_2\} = \frac\{v_1\}\{v_2\}
\end{verbatim}

%%% Line 369
The notation is, for he most part, like the notation in mathematics. Relatively few backslashes and curly braces. If you wish, however, you can use the second, standard LaTeX form. We call streamlined syntax \textit{ETeX} , for \quote{Ergonomic TeX.}

%%% Line 371


%%% Line 375
\section{The Scripta App} \label{the-scripta-app}

\begin{itemize}

%%% Line 377
\item \textbf{Saving} . When editing your document, Scripta saves it every 3 seconds.

%%% Line 379
\item \textbf{Navigation} . To navigate back and forth between documents in Scripta, use the keyboard shortcuts \errorHighlight{[code]?} ctrl-\errorHighlight{[ - can't have space after the bracket}\errorHighlight{extra ]?} and \lstinline|ctrl-| \errorHighlight{extra ]?} or the buttons \textbf{B} and \textbf{F} .

%%% Line 381
\item \textbf{Search and Replace} . To bring up the search and replace panel in the editor, type command-F (mac). To close it, type ESC.

%%% Line 383
\item \textbf{Experimental mode} . Some features are accessible (for now anyway) only in experimental mode. To enter experimental mode, ....

\end{itemize}

%%% Line 385
\section{The Editor} \label{the-editor}

%%% Line 387
The quickest way to open the editor is just to click on a paragraph in the document you are looking at. Amazingly, the editor will open with the same paragraph highlighted.

%%% Line 389
\subsection{Auto-closing} \label{auto-closing}

%%% Line 391
When you type \lstinline|[| , the editor types a following \lstinline|]| and puts the cursr between the left and right brackes. Similarly for parentheses and curly braces \lstinline|\\{\\}|

%%% Line 393
\subsection{Sync} \label{sync}

\begin{itemize}

%%% Line 395
\item Click on rendered text to bring the corresponding source text into focus

%%% Line 397
\item Hold the option key down, select some text, release the option key. They text you selected will be highlighted.

%%% Line 399
\item \textcolor{gray}{Select source text and press\textbf{ ctrl-S} to highlight the corresponding rendered text. --- eliminate this item it is not yet reliable.}

%%% Line 401
\item Press ESC to un-highlight

\end{itemize}

%%% Line 403
\subsection{Commands} \label{commands}

%%% Line 405
Press \textbf{cmd-Z} to undo the last change.

%%% Line 407
\textbf{Search and replace}

%%% Line 409
\compactItem{\textbf{cmd-F} : search (find)}
\compactItem{\textbf{cmd-G} : find next.}
\compactItem{\textbf{cmd-H} : replace all.}
\compactItem{\textbf{ESC} : close editor}

%%% Line 414
\textbf{Move text}

%%% Line 416
\compactItem{\textbf{TAB} : indent selection}
\compactItem{\textbf{shift-TAB} : de-indent selection}

%%% Line 419
\textbf{Move cursor}

%%% Line 421
\compactItem{\textbf{ctrl-A} : beginning of line}
\compactItem{\textbf{ctrl-E} : end of line}
\compactItem{\textbf{ctrl-UPARROW} : go to top of file}
\compactItem{\textbf{ctrl-DOWNARROW} : go to end of file}
\compactItem{\textbf{ctrl-O} : open new line}

%%% Line 427
\textbf{Remove text}

%%% Line 429
\compactItem{\textbf{ctrl-D} : erase next character}
\compactItem{\textbf{ctrl-K} : erase from cursor to end-of-line}

%%% Line 432
\section{Cooperative Editing} \label{cooperative-editing}

%%% Line 434
UNDER CONSTRUCTION

%%% Line 436
\section{Image Manager} \label{image-manager}

%%% Line 438
\subsection{Uploading} \label{uploading}

%%% Line 440
\begin{center}
\includegraphics[width=0.16666666666666666\textwidth]{https://imagedelivery.net/9U-0Y4sEzXlO6BXzTnQnYQ/ff9cfbc7-2143-4f23-289b-be3c110e7e00/public}
\end{center}

%%% Line 443
To place images in your documents, they need to be
accessible from somewhere in the cloud, as is the image on the right and what you find   at \href{https://images.google.com}{}.   Scripta has an image manager that you can use to make a permanent copy of any image AND host on the cloud so that you can use them in Scripta. The image manager keeps everything that you give it in a searchable library, so that you can go back and retrieve images that you added to it a day or a year ago.   To use the image manager, just click on the camera icon in the toolbar.   It will bring up a screen that looks like this:

%%% Line 447
\begin{center}
\includegraphics[width=0.5\textwidth]{https://imagedelivery.net/9U-0Y4sEzXlO6BXzTnQnYQ/f44890b6-e692-458d-a8d0-84ce54929500/public}
\end{center}

%%% Line 450
If your image is on your computer, click the \textbf{Upload} button, navigate to the image, select it, and click \textbf{Open}.
You will see something like the below.   The text "poison dart frog, colors:orange,black,blue" consists of \index{tags} 
that you added yourself.   Tags help you to find images later if you need them.

%%% Line 455
\begin{center}
\includegraphics[width=0.75\textwidth]{https://imagedelivery.net/9U-0Y4sEzXlO6BXzTnQnYQ/a8df913f-eca3-4249-c036-1e3207e26000/public}
\end{center}

%%% Line 458
Now click on \textbf{Save}.   The popup will disappear and a reference for you image will be copied to the clipboard.
Paste it in the document you are working on.   In the editor   will see text something like

%%% Line 461
\begin{verbatim}
| image
https://imagedelivery.net/9U-0Y...
\end{verbatim}

%%% Line 465
In the rendered text, you will see this:

%%% Line 467
\begin{center}
\includegraphics[width=0.51\textwidth]{https://imagedelivery.net/9U-0Y4sEzXlO6BXzTnQnYQ/8aac1f6f-62bc-4dab-282f-1858ac62f800/public}
\end{center}

%%% Line 471
There is another way of uploading an image.   In many web sites, e.g., Google Images you can control-click on an image and then copy it. Or if you have an image in front of you
just do control-5 (Mac) or XXX (PC), then go to the image manager and press control-V to paste it.   Add the tags, click \textbf{Save} and proceed as before.

%%% Line 477
\subsection{Image library} \label{image-library}

%%% Line 479
When you upload an image or paste it as above, it is stored in your Scripta \index{Image library}. To see the image library, option click on the camera icon. When you do, you will see something like this:

%%% Line 481
\begin{center}
\includegraphics[width=0.51\textwidth]{https://imagedelivery.net/9U-0Y4sEzXlO6BXzTnQnYQ/ec0f60c8-57e0-4bbe-02e9-c9dde6a0ab00/public}
\end{center}

%%% Line 484
You can scroll through this library.   If you find an image you need, click on it.   A reference to it will be copied into the clipboard, as before, If you option-click information about the
image will be displayed.

%%% Line 487
Notice text "Filter images" at ehte ,   Type "frog" there.   The image library changes to

%%% Line 490
\begin{center}
\includegraphics[width=0.51\textwidth]{https://imagedelivery.net/9U-0Y4sEzXlO6BXzTnQnYQ/dcc6f608-3f9d-42bc-080e-45794bae1e00/public}
\end{center}

%%% Line 493
Now put "frog poison" in the search field.   You see this:

%%% Line 495
\begin{center}
\includegraphics[width=0.51\textwidth]{https://imagedelivery.net/9U-0Y4sEzXlO6BXzTnQnYQ/7ccc8cfd-74ea-4e5f-969a-86f1c8b79500/public}
\end{center}

%%% Line 498
%%% Line 500
\compactItem{flower --- find images containing "flower"}
\compactItem{flower tree --- find images containing both "flower" AND "tree"}
\compactItem{flower -tree --- find images containing "flower" but NOT "tree"}
\compactItem{@before:7/1/2025 --- find images before a date}
Notice that the images which appear are tagged with both "frog" and "poison."   In general, the filter supports boolean search:

%%% Line 508
\section{Links} \label{links}

%%% Line 510
Make a link to, say, the New York Times with \lstinline|[link NYT http://nytimes.com]|.   This text renders as \href{http://nytimes.com}{NYT}.

%%% Line 512
\subsection{Internal Links} \label{internal-links}

%%% Line 514
Link to another document in Scripta with an \lstinline|ilink| element.

%%% Line 516
\begin{indent}
\href{https://scripta.io/s/jxxcarlson:formal-systems}{Formal Systems} or \href{https://scripta.io/s/jxxcarlson:abstract-art}{Abstract Art}
\end{indent}

%%% Line 519
The first ilink was written using an \lstinline|ilink| (internal link) block:

%%% Line 521
\begin{verbatim}
[ilink Formal Systems jxxcarlson:formal-systems]
\end{verbatim}

%%% Line 525
You can also link to a specific part of a document, provided that it is suitably marked.   We mark the phrase "first partial derivation is stuck" in the Formal Systems document like this:

%%% Line 527
\begin{verbatim}
The [mark first partial derivation is stuck with:stuck]: 
it cannot be continued. 
\end{verbatim}

%%% Line 531
The "mark" for this element is the word "stuck".   We can
link to it from this document by writing

%%% Line 534
\begin{verbatim}
[ilink stuck derivation jxxcarlson:formal-systems#stuck]
\end{verbatim}

%%% Line 537
Then the source text 

%%% Line 539
\begin{verbatim}
Here is an example of a [ilink stuck derivation 
jxxcarlson:formal-systems#stuck]  in a formal system.
\end{verbatim}

%%% Line 543
yields this:

%%% Line 545
\begin{indent}
Here is an example of a \href{https://scripta.io/s/jxxcarlson:formal-systems#stuck}{stuck derivation}   in a formal system.
\end{indent}

%%% Line 548
Try clicking on \textcolor{violet}{stuck derivation} above.   You will be 
transported to the Formal Systems document with the 
marked element highlighted.   To unhighlight, press ESC.   To see
all marked elements, type \textbf{ctrl-opt-=}.

%%% Line 554
\par\vspace{7.5pt}
\hrule{}
\par\vspace{7.5pt}

%%% Line 559
You can make (for example)

\begin{itemize}

%%% Line 561
\item a home page

%%% Line 563
\item a web page with class notes, problem sets, and solutions thereto

\end{itemize}

%%% Line 565
Fill these pages with \lstinline|ilinks| to other Scripta documents, e.g., the problem sets.

%%% Line 567
\subsection{Public Links} \label{public-links}

%%% Line 569
Use the \u{link icon} in the tool strip to copy a link to the current document. This only works if the document is public. Below is the link to this document. You can send it to a friend, colleague or student by email or text so they can look at it. They will be signed in as guest, so they don't need to have a Scripta account. You can also post the link on social media or any substack or web page you have.

%%% Line 571
\begin{verbatim}
https://scripta.io/g/jxxcarlson:welcome-to-scripta
\end{verbatim}

%%% Line 575
\section{Export} \label{export}

%%% Line 577
\subsection{PDF} \label{pdf}

%%% Line 579
\markwith{ecebaadd-7cbb-4bb3-b184-0209bd4af583} .

%%% Line 581
\subsection{LaTeX} \label{latex}

%%% Line 583
\markwith{3b122d77-ed3a-4532-81a1-711186ce9168} . Just click on XXXX.

%%% Line 585
\subsection{Scripata} \label{scripata}

%%% Line 587
\section{Reporting Bugs and Suggestions} \label{reporting-bugs-and-suggestions}

%%% Line 589
Note the \quote{bullhorn} icon on the right-hand edge of this page. Click on it to submit a bug report (issue) or suggestion.

%%% Line 592


%%% Line 596


%%% Line 599


%%% Line 602


%%% Line 605
\bibitem{einstein1905a}
Albert Einstein, ``Über einen die Erzeugung und Verwandlung des Lichtes betreffenden heuristischen Gesichtspunkt'' Annalen der Physik, 17 (1905), pp. 132--148 (On a Heuristic Point of View Concerning the Production and Transformation of Light)

%%% Line 608
\bibitem{einstein1905b}
``Über die von der molekularkinetischen Theorie der Wärme geforderte Bewegung von in ruhenden Flüssigkeiten suspendierten Teilchen'' Annalen der Physik, 17 (1905), pp. 549--560 ( On the Motion of Small Particles Suspended in Stationary Liquids as Required by the Molecular-Kinetic Theory of Heat)

%%% Line 611
\bibitem{einstein1905c}
``Zur Elektrodynamik bewegter Körper'' Annalen der Physik, 17 (1905), pp. 891--921 (On the Electrodynamics of Moving Bodies)

\clearpage

\printindex



\end{document}
