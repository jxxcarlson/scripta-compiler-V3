\documentclass[11pt, oneside]{article}
\usepackage{float}
\usepackage{graphicx}
\usepackage{wrapfig}
\usepackage{xcolor}
\graphicspath{ {image/} }



%% Packages

\usepackage{listings}

%% Standard packages
\usepackage{geometry}
\geometry{letterpaper}
\usepackage{changepage}  % for the adjustwidth environment
\usepackage{graphicx}    % for \includegraphics

%% Index, hyperref
\usepackage[T1]{fontenc}
\usepackage{lmodern}

\usepackage{hyperref}   % load before imakeidx
\usepackage{imakeidx}

%%%%
\usepackage[normalem]{ulem} % for \st
\usepackage{soul}           % for \hl and \sethlcolor
\usepackage{wrapfig}        % for wrapfigure

%% AMS
\usepackage{amssymb}
\usepackage{amsmath}

\usepackage{amscd}

\usepackage{fancyvrb} %% for inline verbatim
\usepackage{makeidx}

%% Chemistry
\usepackage[version=4]{mhchem} % for \ce



%% Commands

\newcommand{\hang}[1]{%
  {%
    \setlength{\leftskip}{1em}%
    \setlength{\hangindent}{1em}%
    \hangafter=1 %
    #1\ \vpace{4}%
  }%
}

\renewcommand{\labelitemi}{\scalebox{0.7}{\textbullet}}

% Dot box = 1em, gap = 1em → total = 2em
\newcommand{\compactItem}[1]{%
  \par
\noindent
  \hangindent=2em \hangafter=1%
  \makebox[1em][l]{\labelitemi}\hspace{1em}#1\par
}

\newcommand{\code}[1]{{\tt #1}}
\newcommand{\ellie}[1]{\href{#1}{Link to Ellie}}
% \newcommand{\image}[3]{\includegraphics[width=3cm]{#1}}

%% width=4truein,keepaspectratio]


% imagecentercaptioned command removed - using standard figure environment instead

\newcommand{\imagecenter}[2]{
   \medskip
   \begin{figure}[htp]
   \centering
    \includegraphics[width=#2]{#1}
    \vglue0pt
    \end{figure}
    \medskip
}

\newcommand{\imagefloat}[4]{
    \begin{wrapfigure}{#4}{#2}
    \includegraphics[width=#2]{#1}
    \caption{#3}
    \end{wrapfigure}
}


\newcommand{\imagefloatright}[3]{
    \begin{wrapfigure}{R}{0.30\textwidth}
    \includegraphics[width=0.30\textwidth]{#1}
    \caption{#2}
    \end{wrapfigure}
}

\newcommand{\hide}[1]{}


\newcommand{\imagefloatleft}[3]{
    \begin{wrapfigure}{L}{0.3-\textwidth}
    \includegraphics[width=0.30\textwidth]{#1}
    \caption{#2}
    \end{wrapfigure}
}
% Font style
\newcommand{\italic}[1]{{\sl #1}}
\newcommand{\strong}[1]{{\bf #1}}
\newcommand{\strike}[1]{\st{#1}}

% Scripta
\newcommand{\ilink}[2]{\href{{https://scripta.io/s/#1}}{#2}}
\newcommand{\markwith}[1]{}
\newcommand{\anchor}[1]{#1}

% Color
\newcommand{\red}[1]{\textcolor{red}{#1}}
\newcommand{\blue}[1]{\textcolor{blue}{#1}}
\newcommand{\violet}[1]{\textcolor{violet}{#1}}
\newcommand{\highlight}[1]{\hl{#1}}
\newcommand{\note}[2]{\textcolor{blue}{#1}{\hl{#1}}}

% WTF?
\newcommand{\remote}[1]{\textcolor{red}{#1}}
\newcommand{\local}[1]{\textcolor{blue}{#1}}

% Unclassified
\newcommand{\subheading}[1]{{\bf #1}\par}
%\newcommand{\term}[1]{{\index{#1}}}
%\newcommand{\termx}[1]{}
\newcommand{\comment}[1]{}
\newcommand{\innertableofcontents}{}


% Special character
\newcommand{\dollarSign}[0]{{\$}}
\newcommand{\backTick}[0]{\`{}}

%% Theorems
\newtheorem{remark}{Remark}
\newtheorem{theorem}{Theorem}
\newtheorem{axiom}{Axiom}
\newtheorem{lemma}{Lemma}
\newtheorem{proposition}{Proposition}
\newtheorem{corollary}{Corollary}
\newtheorem{definition}{Definition}
\newtheorem{example}{Example}
\newtheorem{exercise}{Exercise}
\newtheorem{problem}{Problem}
\newtheorem{exercises}{Exercises}
\newcommand{\bs}[1]{$\backslash$#1}
\newcommand{\texarg}[1]{\{#1\}}


%% Environments
\renewenvironment{quotation}
  {\begin{adjustwidth}{2cm}{} \footnotesize}
  {\end{adjustwidth}}

\def\changemargin#1#2{\list{}{\rightmargin#2\leftmargin#1}\item[]}
\let\endchangemargin=\endlist

\renewenvironment{indent}
  {\begin{adjustwidth}{0.75cm}{}}
  {\end{adjustwidth}}

\newenvironment{box}[1][]{%
  \par\medskip\noindent
  \begin{adjustwidth}{1em}{}
  \ifx\relax#1\relax\else{\bfseries #1}\par\smallskip\fi
}{%
  \end{adjustwidth}\medskip
}

\newcommand{\backtick}{\texttt{\symbol{96}}}

%% NEWCOMMAND

% \definecolor{mypink1}{rgb}{0.858, 0.188, 0.478}
% \definecolor{mypink2}{RGB}{219, 48, 122}
\newcommand{\fontRGB}[4]{
    \definecolor{mycolor}{RGB}{#1, #2, #3}
    \textcolor{mycolor}{#4}
    }

\newcommand{\highlightRGB}[4]{
    \definecolor{mycolor}{RGB}{#1, #2, #3}
    \sethlcolor{mycolor}
    \hl{#4}
     \sethlcolor{yellow}
    }

\newcommand{\gray}[2]{
\definecolor{mygray}{gray}{#1}
\textcolor{mygray}{#2}
}

\newcommand{\white}[1]{\gray{1}[#1]}
\newcommand{\medgray}[1]{\gray{0.5}[#1]}
\newcommand{\black}[1]{\gray{0}[#1]}

% Spacing
\parindent0pt
\parskip5pt

\makeindex[
                          title=Index,
                          columns=2,
                          %% intoc     % include index in the table of contents
                        ]

\begin{document}

\title{Untitled}

\date{}

\author{
test-author
}

\maketitle

\tableofcontents

%%% Line 383
\newcommand{\pd}[2]{\frac{\partial #1}{ \partial #2}}
\newcommand{\pdd}[2]{\frac{\partial^2 #1}{ \partial #2^2}}



%%% Line 3
\begin{center}
\textcolor{red}{(( D R A F T ))}
\end{center}

%%% Line 6
\par\par

%%% Line 8
\begin{indent}
\textit{Scripta is now running a draft of the new Scripta compiler.   It is a work in progress, so some things might be broken or just not look quite right. The final, polished 
version is coming soon.}
\end{indent}

%%% Line 12
\section{What is Scripta?} \label{what-is-scripta}

%%% Line 14
Scripta.io is an app for creating, editing, and publishing
articles with beautiful things like this

%%% Line 17
\begin{center}
\includegraphics[width=0.51\textwidth]{https://imagedelivery.net/9U-0Y4sEzXlO6BXzTnQnYQ/83fdbf6c-79d5-44e7-6ac6-00cdc7785000/public}
\end{center}

%%% Line 21
and this

%%% Line 23
\begin{equation}
\pdd{u}{x} + \pdd{u}{y} + \pdd{u}{z} = \frac{1}{c^2} \pdd{u}{t}
\end{equation}

%%% Line 27
\textit{The equation above governs the transmission of light from sun to flower to the eye of the beholder as in Figure 1}.

%%% Line 31
\par\par\hrule{}\par\par

%%% Line 33
Scripta maked it easy to write short notes with images and equations --- for yourself, your friends and colleagues, or for the class you are teaching.   Here are some examples 
%%% Line 37
 \href{https://scripta.io/s/id-19124175-eb43-4d9d-9f7b-7688f1ee8e1c}{Graphs and Colorings} 
| \href{https://scripta.io/s/id-e54f26b8-4f13-472e-b4ca-68a07879a7cd}{Tikz Example}
| \href{https://scripta.io/s/jxxcarlson:science-stories.1}{Blog} 
| \href{https://scripta.io/s/jxxcarlson:physics-notebook}{Physics Notes}
| \href{https://scripta.io/s/jxxcarlson:abstract-art}{Abstract Art}
of what you can write:

%%% Line 44
\textit{Sharing.} Send a link to a friend or to your students. Or just be retro: writhe link on the blackboard. Another is put the link on your own website.   As an example, the link for this document is

%%% Line 46
\begin{indent}
https://scripta.io/g/jxxcarlson:welcome-to-scripta.1
\end{indent}

%%% Line 49
For more information on the many features of Scripta, take a look at the \href{https://scripta.io/s/id-329ef2b3-321f-4ce8-a830-395724a6daab}{Scripta Manual} or take a look at these videos:

%%% Line 52
\begin{indent}
Video 1 | Video 2 | Video 3 | Video 1 4 Video 5 
\end{indent}

%%% Line 60
\section{The Scripta Markup Language} \label{the-scripta-markup-language}

%%% Line 62
The Scripta markup language   consists of just two things, \index{elements}\textit{elements} and \index{blocks}\textit{blocks}.\footnote{Our aim is to design a langauge that is as simple as possible without losing
expressive power.}

%%% Line 65
\subsection{Elements} \label{elements}

%%% Line 67
Ordinary text, like "Happy New Year" is an element.      So are \index{named elements}\textit{named elements}, which have the form \lstinline|[NAME BODY]|.
Below are some examples: font-style elements, math elements, chemistry elements, and code elements.   There are many more. See XXX

%%% Line 70
\textit{\u{ Font styles.}}      \lstinline|[i Happy New Year]| is an 
\index{italic element}\textit{italic element}.   It renders as \textit{Happy New Year}. 
\lstinline|[b Happy New Year]| is a \index{bold element}\textit{bold element}.
It renders as \textbf{Happy New Year}. 
Elements \index{compose}\textit{compose} so \lstinline|[i [b Happy New Year]]| yields \textit{\textbf{ Happy New Year}}.   

%%% Line 76
It may seem awkward to type \lstinline| [ | then the text of the block, then \lstinline| ] |.   But the Scripta editor helps you: as soon as you type \lstinline| [ | it types \lstinline| ] | and puts the cursor between the 
left and right braces, so you just continue typing.   When you finish typing the body of the element, type \lstinline|]| again and keep going.   Once you learn to do this, you are a speed demon.

%%% Line 79
\textit{\u{ Math.}} Write mathematical formulas, equations, etc. with a \index{math element}\textit{math element}, e.g., \(a^2 + b^2 = c^2\).

%%% Line 81
\begin{indent}
\lstinline|[math a^2 + b^2 = c^2]| \(\quad \to \quad\) \(a^2 + b^2 = c^2\)
\end{indent}

%%% Line 84
The shorthand form \lstinline|[m a^2 + b^2 = c^2]| also works.   If you
want, you can use TeX/LaTex syntax, \lstinline|$a^2 + b^2 = c^2$|.

%%% Line 87
\textit{\u{ Chemistry.}} Write chemical formulas using a 
\textit{chem element}:

%%% Line 90
\begin{indent}
Methane is \lstinline|[chem CH4]| \(\to\) Methane is $\ce{CH4}$
\end{indent}

%%% Line 93
\textit{\u{ Code}.} Write code using a \index{code element}\textit{code element}. 

%%% Line 95
\begin{indent}
\lstinline|[code a^2 + b^ = c^2]| \(\to\) \lstinline|a^2 + b^ = c^2|
\end{indent}

%%% Line 98
You can also enclose
code in backticks, e.g., \backtick{}like this\backtick{}.

%%% Line 103
\subsection{Blocks} \label{blocks}

%%% Line 105
A \index{block}\textit{block} consists of a bunch of contiguous lines of text with an empty line above and below.   Like this:

%%% Line 108
\begin{verbatim}
  -------------
  -------------
  -- block 1 --
  -------------
  -------------
  
  -------------
  -------------
  -- block 2 --
  -------------
  -------------
    
  -------------
  -------------
  -- block 3 --
  -------------
  -------------
\end{verbatim}

%%% Line 128
A paragraph, like the one below, is a kind of block.
Note the mixture of ordinary text and elements like
\lstinline|[red Erik ...]|

%%% Line 132
\begin{verbatim}
[red Erik the Red (Old Norse: Eiríkr rauði)] 
was a Norse mexplorer and settler, best known 
for founding the first Norse colonies in Greenland 
in the late 10th century. We do not know the 
origin of the nickname "Red."  Perhaps the
color of his hair, perhaps is fiery temperament.
(He was exiled from Norway for violence, later
from [blue Iceland] for murder.  
He died in [b [green Greenland.]])
\end{verbatim}

%%% Line 143
Our paragraph renders like this:

%%% Line 145
\begin{indent}
\textcolor{red}{Erik the Red (Old Norse: Eiríkr rauði)} 
was a Norse explorer and settler, best known 
for founding the first Norse colonies in Greenland 
in the late 10th century. We do not know the 
origin of the nickname "Red."   Perhaps the
color of his hair, perhaps is fiery temperament.
(He was exiled from Norway for violence, later
from \textcolor{blue}{Iceland} for murder.   He died in \textbf{\textcolor{green}{ Greenland.}})
\end{indent}

%%% Line 155
Scripta also has \index{named blocks}\textit{named blocks}.   We give four examples:
image blocks, math blocks, chemistry blocks, and code 
blocks.   There are many more, e.g., theorem blocks.   See XXX.

%%% Line 161
\subsubsection*{Image Blocks} \label{image-blocks}

%%% Line 163
\begin{figure}[h]
  \centering
  \includegraphics[width=0.11666666666666667\textwidth]{https://imagedelivery.net/9U-0Y4sEzXlO6BXzTnQnYQ/5c12dfcf-0698-4e0b-733b-f9abf2bfff00/public}
  \caption{Jellyfish}
  \label{fig:jellyfish}
\end{figure}

%%% Line 169
\begin{paragraph}
The tiny image of jellyfish that you see on your 
left is put there using an \index{image block}\textit{image block}.   Click on the image to expand it.   Click on the expanded image to return the view you ser here.   Image blocks have a lot of options.   Lets begin with the simplest case, no option at all:
\end{paragraph}

%%% Line 173
\begin{verbatim}
| image
https://imagedelivery.net/9U-0Y4s...
\end{verbatim}

%%% Line 177
It has the form

%%% Line 179
\begin{verbatim}
HEADER
BODY
\end{verbatim}

%%% Line 184
In this case, the header is the single line \lstinline|| image|.   The body
is the line \lstinline|https://imagedelivery.net...| --- the "internet address" of the image.

%%% Line 187
The image block we used above for the tiny jellyfish is a little more complex:

%%% Line 189
\begin{verbatim}
| image expandable 
| float:left 
| width:70 
| caption:Jellyfish
https://imagedelivery.net/9U-0Y4s...
\end{verbatim}

%%% Line 196
In this case the header consists of four lines, each beginning
with the pipe symbol "|" followed by a space. The first line
of the header is "| image expandable".   Here "expandable" 
is an \index{argument}\textit{argument}.   These are single words that
change the behavior of the image block.   In the case
at hand, it means that if you click on the image it pops to
fill a much larger screen.   Clicking the expanded image returns it to its normal state. There can be more than one argument, but these always come before the properties.

%%% Line 204
The next line of the header is the text "float:left". It defines
a \index{property}\textit{property}   of the image block.   Properties alwyas
have the form ========================================-------------------------------------------------------------------------------------------------------------------------------------------- \lstinline|key:value|.   In this case the key
is "float" and the value associated with that key is "left".
A block with this property hugs the left margin of the 
document and the text below it flows around it.   There is also a \lstinline|float:right| property. The properties set by the next two lines define the width in pixels of image displayed and its caption. 

%%% Line 211
%%% Line 213
 Here is a simple \index{math block}\textit{math block}:
\subsubsection*{Math Blocks} \label{math-blocks}

%%% Line 215
\begin{verbatim}
| math
a^2 + b^2 = c^2
\end{verbatim}

%%% Line 219
It yields this:

%%% Line 221
\begin{equation}
a^2 + b^2 = c^2
\end{equation}

%%% Line 224
We can add a \index{label}\textit{label} property to it:

%%% Line 226
\begin{verbatim}
| math label:pythag
a^2 + b^2 = c^2
\end{verbatim}

%%% Line 231
\begin{equation}
\label{pythag}
a^2 + b^2 = c^2
\end{equation}

%%% Line 234
Any math block that has the label property is automatically
numbered.   The label can be used with the \index{mathref}\textit{mathref}
element to refer back to the equation as in the paragraph
below:

%%% Line 239
\begin{indent}
Equation \eqref{pythag} is the \textit{Pythagorean theorem}, 
where \(a\) and \(b\) are the lengths of the legs of
a right triangle, and \(c\) is the length of the
hypotenuse.
\end{indent}

%%% Line 245
Click on the reference "(1)" to go to equation \eqref{pythag}.   Press
ESC to return to where you were. We made the live reference
(1) by writing \lstinline|[mathref pythag]|.   

%%% Line 249
\textit{For LaTeX users:} As a convenience, Scripta provides \index{equation blocks}\textit{equation blocks} and \index{eqref elements}\textit{eqref elements}.   These are 
synomyms for math blocks and mathref elements. They corresond to equation environments and eqref macros in LaTeX.

%%% Line 254
\subsubsection*{Chemistry Blocks} \label{chemistry-blocks}

%%% Line 256
Chemical formulas and reactions can be rendered using \index{chem blocks}\textit{chem blocks}.   Burning hydrogen in oxygen takes place via the reaction 

%%% Line 258
\[\ce{2H2 + O2 -> 2H2O + Energy}\]

%%% Line 261
To produce the text above, we used

%%% Line 263
\begin{verbatim}
  | chem 
  2H2 + O2 -> 2H2O + Energy  
\end{verbatim}

%%% Line 267
See Martin Hensel's \href{https://texdoc.org/serve/mhchem/0}{mhchem bundle} for a comprehensive discussion of what you can do with chem elements and blocks.

%%% Line 269
\subsubsection*{Code Blocks} \label{code-blocks}

%%% Line 271
Below is an example of a \index{code block}\textit{code block}.   Click on it
to reveal it in the editor.   The header of the block is

%%% Line 274
\begin{verbatim}
| code python linenumbers:yes
\end{verbatim}

%%% Line 277
Because of the argument \lstinline|python|, the rendered text 
is shown with Python syntax highlighting. The property
\lstinline|linenumbers:yes| tells Scripta to format the code
with line numbers.

%%% Line 283
\begin{verbatim}
  def is_prime(n):
      if n < 2:
          return False
      for k in range(2, int(n**0.5) + 1):
          if n % k == 0:
              return False
      return True
  
  
  def print_primes_up_to(N):
      for p in range(2, N + 1):
          if is_prime(p):
              print(p)
  
  
  # example usage
  N = 100
  print_primes_up_to(N)
\end{verbatim}

%%% Line 310
Because the text of the code block consists of several
paragraphs, we indent it by selecting it all and pressing TAB.
The empty lines become short lines consisting of 
two spaces.   There are no empty lines. 

%%% Line 315
Click anywhere in the rendered code block to reveal it 
in the editor. There you can see the indentation structure
Because the block has empty lines, the entire block is indented by spaces. XXX(Clarity of text???)

%%% Line 319
Tip: XXX

%%% Line 321
\section{The Scripta App} \label{the-scripta-app}

%%% Line 323
\begin{center}
\includegraphics[width=0.3333333333333333\textwidth]{https://imagedelivery.net/9U-0Y4sEzXlO6BXzTnQnYQ/f3c5c902-809c-4629-7015-a4327103a000/public}
\end{center}

%%% Line 326
On the left you see a thumbnail preview of the Scripta.io
app.   This document is being edited.   You see the source
text on the left, the rendered text on the right. 
Click on this image to make it expand.   Click on it again to restore the previous view.   Do this now.   Do it several times to ingrain it in your muscle memory.

%%% Line 331
One of the main features of Scripta is that as you type new source text or make edits to the source text, those changes
will be rendered almost instantly in the rendered text window. See (XXX VIDEO)

%%% Line 335
To edit a document, click on the Pen icon which you will see above the rendered text. You can also click on any part of the rendered text.\footnote{There are some exceptions, e.g. ilink elements XXX.} Click on the $\boxtimes$ to leave the editor.

%%% Line 337
There is a lot more to the Scripta app. See XXX. You can, for example, export Scripta documents to PDF and then print it (XXX SEE VIDEO). Just click on the printer icon and follow directions. You can also export any Scripta document to LaTeX.\footnote{There are some features XXX that
are not compatible with LaTeX, so these are not exported.   See XXX for a complete list of these.}
\textcolor{gray}{By adding style files, you can prepare it for publication by
a journal that requires submission in LaTeX in a particular format. XXX: to be investigated}
 
 

%%% Line 345


%%% Line 351
\section{Your First Document} \label{your-first-document}

%%% Line 354
\textit{Take
a look at this short video (XXX). It will show you
how to create your first document, write equations,
and place images, and more.}

%%% Line 361
\section{Features} \label{features}

\begin{itemize}

%%% Line 363
\item \textbf{No set-up}.   Create a new document with the click of a button. Then start typing.

%%% Line 365
\item \textbf{Web-based}. Scripta produces web pages, not PDFs, although you can do that too.

%%% Line 367
\item \textbf{Instant rendering}. As you type, the web pageyou are working on --- text, images, equations, cross references, table of contents, etc. --- is updated.   All in real time.

%%% Line 369
\item \textbf{Modern editor}.   Click on a word in the rendered text.It will be highlighted in the source text. Selectsome source text, type ctrl-S. The corresponding rendered text will be highlighted. Type ctrl-1 to remove thehighlight.

%%% Line 372
\item \textbf{Available everywhere, anytime}. Your writing is available to you, your students, and your colleages anywhere there is an internet connection.   Readable on smart phones, tablets, laptops, and desktop computers.

%%% Line 374
\item \textbf{No vendor lock-in}.   Philosophically, we are opposed to vendor lock-in.   It is wrong. Practically, we provide tools for you to export both individual documents and yourentire corpus of documents.   See XXX.   Additionally,the Scripta compiler, which is open source, is avaiable on \href{https://github.com/jxxcarlson/scripta-compiler-v2}{Github}.

%%% Line 376
\item \textbf{Integrated image uploader}. Click on the camera icon for placing images in your document. Paste the image into the uploader, then paste the resulting image block in the editor.

\end{itemize}

%%% Line 378
\begin{center}
\includegraphics[width=0.51\textwidth]{https://imagedelivery.net/9U-0Y4sEzXlO6BXzTnQnYQ/5111dd45-daa1-4183-315a-7bec8802a700/public}
\end{center}

%%% Line 383


%%% Line 387
\par\par\hrule{}

%%% Line 389
\textit{Below is the "Endnotes" section.   This is where the footnote text is displayed.   To place a footnote after
some text, just write\lstinline|[footnote Blah blah blah]|}\footnote{Blah blah blah}.

%%% Line 392


%%% Line 394
\par\par\hrule{}

%%% Line 396
\textit{Below you will find an index of terms mentioned in this documentt.   Click on a word or phrase in the index. The app will scroll up to where that term is mentioned and highlight the paragraph in which it appears.   Press ESC to clear the highlight.   Press option-ESC (alt-ESC) to return to the index.}

%%% Line 398


\clearpage

\printindex



\end{document}
