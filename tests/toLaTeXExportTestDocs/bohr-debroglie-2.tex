\documentclass[11pt, oneside]{article}



%% Packages

\usepackage{listings}

%% Standard packages
\usepackage{geometry}
\geometry{letterpaper}
\usepackage{changepage}  % for the adjustwidth environment
\usepackage{graphicx}
\graphicspath{{/Users/carlson/dev/elm-work/scripta/pdfServer2/image/}}    % for \includegraphics

%% Index, hyperref
\usepackage[T1]{fontenc}
\usepackage{lmodern}

\usepackage{hyperref}   % load before imakeidx
\usepackage{imakeidx}

%%%%
\usepackage[normalem]{ulem} % for \st
\usepackage{soul}           % for \hl and \sethlcolor
\usepackage{wrapfig}        % for wrapfigure

%% AMS
\usepackage{amssymb}
\usepackage{amsmath}

\usepackage{amscd}

\usepackage{fancyvrb} %% for inline verbatim
\usepackage{makeidx}

%% Chemistry
\usepackage[version=4]{mhchem} % for \ce



%% Commands

\newcommand{\hang}[1]{%
  {%
    \setlength{\leftskip}{1em}%
    \setlength{\hangindent}{1em}%
    \hangafter=1 %
    #1\ \vpace{4}%
  }%
}

\renewcommand{\labelitemi}{\scalebox{0.7}{\textbullet}}

% Dot box = 1em, gap = 1em → total = 2em
\newcommand{\compactItem}[1]{%
  \par
\noindent
  \hangindent=2em \hangafter=1%
  \makebox[1em][l]{\labelitemi}\hspace{1em}#1\par
}

\newcommand{\code}[1]{{\tt #1}}
\newcommand{\ellie}[1]{\href{#1}{Link to Ellie}}
% \newcommand{\image}[3]{\includegraphics[width=3cm]{#1}}

%% width=4truein,keepaspectratio]


% imagecentercaptioned command removed - using standard figure environment instead

\newcommand{\imagecenter}[2]{
   \medskip
   \begin{figure}[htp]
   \centering
    \includegraphics[width=#2]{#1}
    \vglue0pt
    \end{figure}
    \medskip
}

\newcommand{\imagefloat}[4]{
    \begin{wrapfigure}{#4}{#2}
    \includegraphics[width=#2]{#1}
    \caption{#3}
    \end{wrapfigure}
}


\newcommand{\imagefloatright}[3]{
    \begin{wrapfigure}{R}{0.30\textwidth}
    \includegraphics[width=0.30\textwidth]{#1}
    \caption{#2}
    \end{wrapfigure}
}

\newcommand{\hide}[1]{}


\newcommand{\imagefloatleft}[3]{
    \begin{wrapfigure}{L}{0.3-\textwidth}
    \includegraphics[width=0.30\textwidth]{#1}
    \caption{#2}
    \end{wrapfigure}
}
% Font style
\newcommand{\italic}[1]{{\sl #1}}
\newcommand{\strong}[1]{{\bf #1}}
\newcommand{\strike}[1]{\st{#1}}

% Scripta
\newcommand{\ilink}[2]{\href{{https://scripta.io/s/#1}}{#2}}
\newcommand{\markwith}[1]{}
\newcommand{\anchor}[1]{#1}

% Color
\newcommand{\red}[1]{\textcolor{red}{#1}}
\newcommand{\blue}[1]{\textcolor{blue}{#1}}
\newcommand{\violet}[1]{\textcolor{violet}{#1}}
\newcommand{\highlight}[1]{\hl{#1}}
\newcommand{\note}[2]{\textcolor{blue}{#1}{\hl{#1}}}

% WTF?
\newcommand{\remote}[1]{\textcolor{red}{#1}}
\newcommand{\local}[1]{\textcolor{blue}{#1}}

% Unclassified
\newcommand{\subheading}[1]{{\bf #1}\par}
%\newcommand{\term}[1]{{\index{#1}}}
%\newcommand{\termx}[1]{}
\newcommand{\comment}[1]{}
\newcommand{\innertableofcontents}{}


% Special character
\newcommand{\dollarSign}[0]{{\$}}
\newcommand{\backTick}[0]{\`{}}

%% Theorems
\newtheorem{remark}{Remark}
\newtheorem{theorem}{Theorem}
\newtheorem{axiom}{Axiom}
\newtheorem{lemma}{Lemma}
\newtheorem{proposition}{Proposition}
\newtheorem{corollary}{Corollary}
\newtheorem{definition}{Definition}
\newtheorem{example}{Example}
\newtheorem{exercise}{Exercise}
\newtheorem{problem}{Problem}
\newtheorem{exercises}{Exercises}
\newcommand{\bs}[1]{$\backslash$#1}
\newcommand{\texarg}[1]{\{#1\}}


%% Environments
\renewenvironment{quotation}
  {\begin{adjustwidth}{2cm}{} \footnotesize}
  {\end{adjustwidth}}

\def\changemargin#1#2{\list{}{\rightmargin#2\leftmargin#1}\item[]}
\let\endchangemargin=\endlist

\renewenvironment{indent}
  {\begin{adjustwidth}{0.75cm}{}}
  {\end{adjustwidth}}

\newenvironment{box}[1][]{%
  \par\medskip\noindent
  \begin{adjustwidth}{1em}{}
  \ifx\relax#1\relax\else{\bfseries #1}\par\smallskip\fi
}{%
  \end{adjustwidth}\medskip
}

\newcommand{\backtick}{\texttt{\symbol{96}}}

%% NEWCOMMAND

% \definecolor{mypink1}{rgb}{0.858, 0.188, 0.478}
% \definecolor{mypink2}{RGB}{219, 48, 122}
\newcommand{\fontRGB}[4]{
    \definecolor{mycolor}{RGB}{#1, #2, #3}
    \textcolor{mycolor}{#4}
    }

\newcommand{\highlightRGB}[4]{
    \definecolor{mycolor}{RGB}{#1, #2, #3}
    \sethlcolor{mycolor}
    \hl{#4}
     \sethlcolor{yellow}
    }

\newcommand{\gray}[2]{
\definecolor{mygray}{gray}{#1}
\textcolor{mygray}{#2}
}

\newcommand{\white}[1]{\gray{1}[#1]}
\newcommand{\medgray}[1]{\gray{0.5}[#1]}
\newcommand{\black}[1]{\gray{0}[#1]}

% Spacing
\parindent0pt
\parskip5pt

\makeindex[
                          title=Index,
                          columns=2,
                          %% intoc     % include index in the table of contents
                        ]

\begin{document}

\title{Untitled}

\date{}

\author{
test-author
}

\maketitle

\tableofcontents



%%% Line 3


%%% Line 7


%%% Line 16
\par\vspace{7.5pt}

%%% Line 18
\textit{Discussion from Weinberg, Introduction to Modern Physics, p. 125}

%%% Line 20
De Broglie thought that if an electromagnetic wave can sometimes
be interpreted as a stream of particles --- photons --- then a particle
like the electron might also be viewed as a wave.   Recall that
the energy of a photon is given by \(E = \hbar\omega\) and that
that its momentum is given by \(\mathbf{p} = \hbar \mathbf{k}\).
These are the Planck-Einstein and de Broglie relations, respectively.   

%%% Line 27
\section{Electron waves} \label{electron-waves}

%%% Line 29
De Broglie suggested that an electron consists of a wave packet
built of plane waves 

%%% Line 32
\begin{equation}
\label{electron-wave}
\psi_{\mathbf{p}} \propto \exp [i\mathbf{p}\cdot\mathbf{x}/\hbar - iE(\mathbf{p})t/\hbar ]
\end{equation}

%%% Line 35
where

%%% Line 37
\begin{equation}
\label{energy-formula}
E(\mathbf{p})  = \sqrt{m_e^2c^4 + \mathbf{p}^2c^2}
\end{equation}

%%% Line 41
The wave packet is given by the Fourier integral

%%% Line 43
\begin{equation}
\psi(\mathbf{x},t) = \int d^3p\, g(\mathbf{p}) \psi_\mathbf{p}(\mathbf{x}, t)
\end{equation}

%%% Line 46
Consider the Taylor series for \(E(\mathbf{p})\) expanded around
a value \(\mathbf{P}\) around which the momementum is concentrated:

%%% Line 49
\begin{equation}
E(\mathbf{p}) = E(\mathbf{P}) + \mathbf{V}\cdot(\mathbf{p} - \mathbf{P}) + \cdots
\end{equation}

%%% Line 52
Here 

%%% Line 54
\begin{equation}
\mathbf{V}_i = \frac{\partial{E}(\mathbf{p})}{\partial p_i}\Big\vert_{\mathbf{p} = \mathbf{P}}
\end{equation}

%%% Line 58
Differentiating \eqref{energy-formula} with respect to \(p_i\)
we obtain

%%% Line 61
\begin{align}
\frac{\partial{E}(\mathbf{p})}{\partial p_i}\Big\vert_{\mathbf{p} = \mathbf{P}} &= (1/2)(m_e^2c^4 + (\mathbf{p}^2c^2)^{-1/2}(2p_ic^2 )\\
&= \frac{p_ic^2}{E(\mathbf{P})}
\end{align}

%%% Line 66
From the proportionality relations 
\(\mathbf{p}(v) = m\gamma\mathbf{v}\)
and \(E(\mathbf{v}) = mc^2\gamma\), it follows that

%%% Line 70
\begin{equation}
\mathbf{V_i} = \frac{\partial{E}(\mathbf{p})}{\partial p_i}\Big\vert_{\mathbf{p} = \mathbf{P}}= v_i
\end{equation}

%%% Line 74
so the group velocity of the wave packet representing the particle
and the classical velocity of the particle are the same.

%%% Line 79
\section{The Bohr Atom} \label{the-bohr-atom}

%%% Line 81
Suppose that the electron wave is confined to a circular orbit
around the proton.   The requirement of continuity is the requirement
that the orbit consists of a whole number of wavelengths, so that 
\(2\pi r = n\lambda\).   According to \eqref{electron-wave}, 
\(\lambda = h/p\), so we have

%%% Line 87
\begin{equation}
\label{momentum-quantization}
pr = n\hbar
\end{equation}

%%% Line 90
For non-relativistic electrons, \(p = m_ev\), so 
\eqref{momentum-quantization} gives Bohr's result,
\(m_ev_nr_n = n\hbar\).
This result, combined with the classical relations

%%% Line 95
\begin{align}
\label{balance-e}
\frac{Ze^2}{r_n^2} &= \frac{m_ev_n^2}{r_n}\\
E_n &= \frac{1}{2} m_e v_n^2 - \frac{Ze^2}{r_n}
\end{align}

%%% Line 100
yield Bohr's formula

%%% Line 102
\begin{equation}
\label{electron-E-quantized}
E_n = -\frac{1}{2} \frac{Z^2e^4m_e}{n^2\hbar^2}
\end{equation}

%%% Line 105
While nothing new is found, Bohr's formula now rests
upon more fundamental assumptions.

\clearpage

\printindex



\end{document}
