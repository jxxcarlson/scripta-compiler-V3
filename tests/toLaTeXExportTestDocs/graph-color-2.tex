\documentclass[11pt, oneside]{article}
\usepackage{tikz-cd}


%% Packages

\usepackage{listings}

%% Standard packages
\usepackage{geometry}
\geometry{letterpaper}
\usepackage{changepage}  % for the adjustwidth environment
\usepackage{graphicx}
\graphicspath{{/Users/carlson/dev/elm-work/scripta/pdfServer2/image/}}    % for \includegraphics

%% Index, hyperref
\usepackage[T1]{fontenc}
\usepackage{lmodern}

\usepackage{hyperref}   % load before imakeidx
\usepackage{imakeidx}

%%%%
\usepackage[normalem]{ulem} % for \st
\usepackage{soul}           % for \hl and \sethlcolor
\usepackage{wrapfig}        % for wrapfigure

%% AMS
\usepackage{amssymb}
\usepackage{amsmath}

\usepackage{amscd}

\usepackage{fancyvrb} %% for inline verbatim
\usepackage{makeidx}

%% Chemistry
\usepackage[version=4]{mhchem} % for \ce



%% Commands

\newcommand{\hang}[1]{%
  {%
    \setlength{\leftskip}{1em}%
    \setlength{\hangindent}{1em}%
    \hangafter=1 %
    #1\ \vpace{4}%
  }%
}

\renewcommand{\labelitemi}{\scalebox{0.7}{\textbullet}}

% Dot box = 1em, gap = 1em → total = 2em
\newcommand{\compactItem}[1]{%
  \par
\noindent
  \hangindent=2em \hangafter=1%
  \makebox[1em][l]{\labelitemi}\hspace{1em}#1\par
}

\newcommand{\code}[1]{{\tt #1}}
\newcommand{\ellie}[1]{\href{#1}{Link to Ellie}}
% \newcommand{\image}[3]{\includegraphics[width=3cm]{#1}}

%% width=4truein,keepaspectratio]


% imagecentercaptioned command removed - using standard figure environment instead

\newcommand{\imagecenter}[2]{
   \medskip
   \begin{figure}[htp]
   \centering
    \includegraphics[width=#2]{#1}
    \vglue0pt
    \end{figure}
    \medskip
}

\newcommand{\imagefloat}[4]{
    \begin{wrapfigure}{#4}{#2}
    \includegraphics[width=#2]{#1}
    \caption{#3}
    \end{wrapfigure}
}


\newcommand{\imagefloatright}[3]{
    \begin{wrapfigure}{R}{0.30\textwidth}
    \includegraphics[width=0.30\textwidth]{#1}
    \caption{#2}
    \end{wrapfigure}
}

\newcommand{\hide}[1]{}


\newcommand{\imagefloatleft}[3]{
    \begin{wrapfigure}{L}{0.3-\textwidth}
    \includegraphics[width=0.30\textwidth]{#1}
    \caption{#2}
    \end{wrapfigure}
}
% Font style
\newcommand{\italic}[1]{{\sl #1}}
\newcommand{\strong}[1]{{\bf #1}}
\newcommand{\strike}[1]{\st{#1}}

% Scripta
\newcommand{\ilink}[2]{\href{{https://scripta.io/s/#1}}{#2}}
\newcommand{\markwith}[1]{}
\newcommand{\anchor}[1]{#1}

% Color
\newcommand{\red}[1]{\textcolor{red}{#1}}
\newcommand{\blue}[1]{\textcolor{blue}{#1}}
\newcommand{\violet}[1]{\textcolor{violet}{#1}}
\newcommand{\highlight}[1]{\hl{#1}}
\newcommand{\note}[2]{\textcolor{blue}{#1}{\hl{#1}}}

% WTF?
\newcommand{\remote}[1]{\textcolor{red}{#1}}
\newcommand{\local}[1]{\textcolor{blue}{#1}}

% Unclassified
\newcommand{\subheading}[1]{{\bf #1}\par}
%\newcommand{\term}[1]{{\index{#1}}}
%\newcommand{\termx}[1]{}
\newcommand{\comment}[1]{}
\newcommand{\innertableofcontents}{}


% Special character
\newcommand{\dollarSign}[0]{{\$}}
\newcommand{\backTick}[0]{\`{}}

%% Theorems
\newtheorem{remark}{Remark}
\newtheorem{theorem}{Theorem}
\newtheorem{axiom}{Axiom}
\newtheorem{lemma}{Lemma}
\newtheorem{proposition}{Proposition}
\newtheorem{corollary}{Corollary}
\newtheorem{definition}{Definition}
\newtheorem{example}{Example}
\newtheorem{exercise}{Exercise}
\newtheorem{problem}{Problem}
\newtheorem{exercises}{Exercises}
\newcommand{\bs}[1]{$\backslash$#1}
\newcommand{\texarg}[1]{\{#1\}}


%% Environments
\renewenvironment{quotation}
  {\begin{adjustwidth}{2cm}{} \footnotesize}
  {\end{adjustwidth}}

\def\changemargin#1#2{\list{}{\rightmargin#2\leftmargin#1}\item[]}
\let\endchangemargin=\endlist

\renewenvironment{indent}
  {\begin{adjustwidth}{0.75cm}{}}
  {\end{adjustwidth}}

\newenvironment{box}[1][]{%
  \par\medskip\noindent
  \begin{adjustwidth}{1em}{}
  \ifx\relax#1\relax\else{\bfseries #1}\par\smallskip\fi
}{%
  \end{adjustwidth}\medskip
}

\newcommand{\backtick}{\texttt{\symbol{96}}}

%% NEWCOMMAND

% \definecolor{mypink1}{rgb}{0.858, 0.188, 0.478}
% \definecolor{mypink2}{RGB}{219, 48, 122}
\newcommand{\fontRGB}[4]{
    \definecolor{mycolor}{RGB}{#1, #2, #3}
    \textcolor{mycolor}{#4}
    }

\newcommand{\highlightRGB}[4]{
    \definecolor{mycolor}{RGB}{#1, #2, #3}
    \sethlcolor{mycolor}
    \hl{#4}
     \sethlcolor{yellow}
    }

\newcommand{\gray}[2]{
\definecolor{mygray}{gray}{#1}
\textcolor{mygray}{#2}
}

\newcommand{\white}[1]{\gray{1}[#1]}
\newcommand{\medgray}[1]{\gray{0.5}[#1]}
\newcommand{\black}[1]{\gray{0}[#1]}

% Spacing
\parindent0pt
\parskip5pt

\makeindex[
                          title=Index,
                          columns=2,
                          %% intoc     % include index in the table of contents
                        ]

\begin{document}

\title{Untitled}

\date{}

\author{
test-author
}

\maketitle

\tableofcontents

%%% Line 9
\newcommand{\set}[1]{\{\ #1 \ \}}
\newcommand{\Set}{\mathop{\underline{\text{Set}}}}
\newcommand{\bN}{\mathbb{N}}
\newcommand{\Hom}{\op{Hom}}
\newcommand{\op}[1]{\mathop{\text{#1}}}

%%% Line 1


%%% Line 5


%%% Line 9


%%% Line 17


%%% Line 23


%%% Line 25
\section{Graphs as Functors} \label{graphs-as-functors}

%%% Line 28


%%% Line 30
Let \(\Gamma\) be the category with objects \(E\) and \(V\)
and with arrows \(s, t: E \to V\), where \(s\) picks out
the source (tail) of an arrow and \(t\) picks out the 
target (head):

%%% Line 35
\begin{equation}
\Gamma = (E \xrightarrow{s, t} V)
\end{equation}

%%% Line 38
A \index{graph}\textit{graph} is a functor \(G: \Gamma \to \Set\).   Consider, 
for example, the graph in Figure \ref{abcd-y-graph} below:

%%% Line 42
%%
\[\begin{tikzcd}
\end{tikzcd}\]

%%% Line 57
In this case, \(G\) is the functor wiiith 

%%% Line 59
\begin{align}
G(V) &= \set{a, b, c, d}\\
G(E) &= \set{ab, bc, bd}
\end{align}

%%% Line 63
where \(ab\) is the edge connecting \(a\) to \(b\), etc.   
Since \(G\) is a functor, we have

%%% Line 66
\begin{align}
G(s)&: G(E) \to G(V)\\
G(t)&: G(E) \to G(V)
\end{align}

%%% Line 70
Thus \(G(s)(ab) = a\), \(G(t)(ab) = b\), etc.

%%% Line 72
Let \(\op{Graph} = [\Gamma, \Set]\) be the (functor) category of directed graphs.   Then a morphism of graphs is a natural
transformation of functors \(\eta\).   Thus, if \(G\) and \(G'\) are graphs, and \(\eta : G \to G'\) is a morphism, then 
\(\eta_{E} : G(E) \to G'(E)\) maps edges of the first graph
to edges of the second and \(\eta_V : G(V) \to G'(V)\)
maps vertices to vertices.      The fact that \(\eta\) is a natural transformation, given by the equations or by the 
commutative diagram or Figure   \ref{morphism-graph},

%%% Line 79
\begin{equation}
\label{functoriality}
\eta_V \circ G(s) = G'(s) \circ \eta_{E} \\
\eta_V \circ G(t) = G'(t) \circ \eta_{E} \\
\end{equation}

%%% Line 83
means that the given assignment of edges and vertices
preserve incidence relations.

%%% Line 87
%%
\[\begin{tikzcd}
\end{tikzcd}\]

%%% Line 104
As an example, consider the
morphism pictured in the Figure \ref{graph-morphism} below.
Then \(\eta_V(b) = 2\), \(\eta_V(c) = 3\), 
\(\eta_{E}(bc) = 23\), etc.

%%% Line 109
% https://q.uiver.app/?q=WzAsNyxbMCwxLCJhIl0sWzIsMSwiYiJdLFswLDQsIjEiXSxbMiw0LCIyIl0sWzQsNCwiMyJdLFs0LDAsImMiXSxbNCwyLCJkIl0sWzAsMV0sWzIsM10sWzMsNF0sWzAsMiwiXFxldGFfYSIsMSx7InN0eWxlIjp7ImJvZHkiOnsibmFtZSI6ImRhc2hlZCJ9fX1dLFsxLDMsIlxcZXRhX2IiLDEseyJzdHlsZSI6eyJib2R5Ijp7Im5hbWUiOiJkYXNoZWQifX19XSxbMSw1XSxbNSw0LCJcXGV0YV9jIiwxLHsiY3VydmUiOi01LCJzdHlsZSI6eyJib2R5Ijp7Im5hbWUiOiJkYXNoZWQifX19XSxbMSw2XSxbNiw0LCJcXGV0YV9kIiwxLHsic3R5bGUiOnsiYm9keSI6eyJuYW1lIjoiZGFzaGVkIn19fV1d
\[\begin{tikzcd}
	&&&& c \\
	a && b \\
	&&&& d \\
	\\
	1 && 2 && 3
	\arrow[from=2-1, to=2-3]
	\arrow[from=5-1, to=5-3]
	\arrow[from=5-3, to=5-5]
	\arrow["{\eta_a}"{description}, dashed, from=2-1, to=5-1]
	\arrow["{\eta_b}"{description}, dashed, from=2-3, to=5-3]
	\arrow[from=2-3, to=1-5]
	\arrow["{\eta_c}"{description}, curve={height=-30pt}, dashed, from=1-5, to=5-5]
	\arrow[from=2-3, to=3-5]
	\arrow["{\eta_d}"{description}, dashed, from=3-5, to=5-5]
\end{tikzcd}\]

%%% Line 136
\section{Colorings} \label{colorings}

%%% Line 139
A \index{coloring}\textit{coloring} of a graph is a function \(c\) from its vertices to a set \(C\), the set of colors, such that vertices 
connected by an edge have different colors:

%%% Line 142
\begin{equation}
\forall e \in G(E): c(G(s)(e)) \ne c(G(t)(e))
\end{equation}

%%% Line 145
If \(C\) has \(n\) elements, then \(c\) is an \(n\)-coloring. As an example, the graph in Figure \ref{abcd-y-graph} has a 2-coloring
with red and green, as indicated in Figure \ref{fig-colored-graph}.

%%% Line 148
% https://q.uiver.app/?q=WzAsNCxbMCwxLCJhLCBSIl0sWzIsMSwiYiwgRyJdLFs0LDAsImMsIFIiXSxbNCwyLCJkLCBSIl0sWzAsMV0sWzEsMl0sWzEsM11d
\[\begin{tikzcd}
	&&&& {c, R} \\
	{a, R} && {b, G} \\
	&&&& {d, R}
	\arrow[from=2-1, to=2-3]
	\arrow[from=2-3, to=1-5]
	\arrow[from=2-3, to=3-5]
\end{tikzcd}\]

%%% Line 165
Let \(C_n : \op{Graph} \to \Set\) be the functor which assigns to a graph the corresponding set of \(n\)-colorings.
(To make this statement precise, we assume that colors are 
taken from a fixed but infinite set of color names.   
The 
set \(\bN\) of natural numbers will do.) We say that a graph
\(G\) is \index{n-colorable}\textit{n-colorable} if \(C_n(G)\) is non-empty.   A
famous theorem asserts that every planar graph is 4-colorable.

%%% Line 173
An \(n\)-coloring \(c\) is a 
function \(c: G \to \set{1, \ldots, n}\).   If \(\phi : G' \to G\)
is a morphism, then \(c \circ \phi\) is an \(n\)-coloring of \(G'\).

%%% Line 177
\begin{problem}
What happens to the colorability of the graph in Figure \ref{fig-colored-graph} if
we add the vertex \(cd\)?
\end{problem}

%%% Line 181
\section{Complete graphs} \label{complete-graphs}

%%% Line 184
A complete graph is one in which any two vertices are 
connected by an edge, as in Figure \eqref{complete-graph-4}.

%%% Line 187
% https://q.uiver.app/?q=WzAsNCxbMCwwLCJBIl0sWzIsMCwiRCJdLFswLDIsIkIiXSxbMiwyLCJDIl0sWzAsMl0sWzIsM10sWzMsMV0sWzEsMF0sWzAsM10sWzEsMl1d
\[\begin{tikzcd}
	A && D \\
	\\
	B && C
	\arrow[from=1-1, to=3-1]
	\arrow[from=3-1, to=3-3]
	\arrow[from=3-3, to=1-3]
	\arrow[from=1-3, to=1-1]
	\arrow[from=1-1, to=3-3]
	\arrow[from=1-3, to=3-1]
\end{tikzcd}\]

%%% Line 207
Let \(K_n\) be the graph with vertices \(\set{1, \ldots, n}\)
and with edges \(ij\), for all \(i < j\).   Thus \(K_4\) is isomorphic
to the graph of Figure \ref{complete-graph-4}. 

%%% Line 211
Consider the contravariant function \(\text{Hom}(-, K_n)\).
We claim that
there is a natural bijection of sets

%%% Line 215
\begin{equation}
C_n(G) \to \text{Hom}(G, K_n)
\end{equation}

%%% Line 219
so that \(C_n(-)\) and \(\text{Hom}(-, K_n)\) are naturally isomorphic 
functors. It follows that \(C_n\) is a \index{representable functor}\textit{representable functor}.

\clearpage

\printindex



\end{document}
